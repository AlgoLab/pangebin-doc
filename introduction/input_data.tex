\subsection{Assembly plasmid binning input data}

\begin{definition}{Contig set}{contig_set}
  Let \Contigs{} be the contig set.
  A contig is a sequence resulting from a read assembly process.
\end{definition}

Each contig is associated with three properties: its coverage, its plasmidness and a list of GC scores depending on GC content intervals.

In the following, \(c \in \Contigs{}\) is a contig.

\begin{definition}{Contig coverage}{contig_coverage}
  By \(\cov{c} \in \Reals_{>0}\) we denote the coverage of \(c\).
  It corresponds to the average base coverage of the contig normalized by the median coverage of the contigs in the whole assembly.
\end{definition}

\begin{definition}{Contig GC score}{contig_GC_score}
  \begin{newfeatbox}
    We do not normalize the probabilities by the sum of the \(P(b|n,l)\) over all \(b \in K\) as in~\cite{manePlasBinflowFlowbasedMILP2023}. We normalize them according to the maximum among them.
  \end{newfeatbox}
  Let \(c \in \Contigs{}\) be a contig and \(b\) be a GC content interval.
  By \(\gcscore{c}{b} \in [-1, 1]\) we denote the GC score of the contig \(c\) according to the GC content interval \(b\):
  \[
    \gcscore{c}{b} = 2 \frac{\Pr(n|b,l)\Pr(b)}{\max\limits_{b' \in K}\Set*{\Pr(n|b',l)\Pr(b')}} - 1
  \]
  Where \(K\) is the set of GC content intervals, \(n\) is the number of GC in the contig of length \(l\), and \(\Pr(n|b,l)\) is calculated as described in~\cite{manePlasBinflowFlowbasedMILP2023}, Section 2.5.1.

  \begin{fixmebox}
    Adapt the correction constraints according to the new score definition.
  \end{fixmebox}

  \begin{questionbox}
    For both performance and interpretation issue, we should not have a negative GC score.
  \end{questionbox}
\end{definition}

\begin{definition}{Contig plasmidness}{contig_plasmidness}
  Let \(c \in \Contigs{}\) be a contig.
  By \(\plm{c} \in [-1, 1]\) we denote its plasmidness score.
  A plasmid sequence classifier can help to provid these scores.
\end{definition}