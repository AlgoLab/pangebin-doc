\subsection{Contigs in the assembly-consensus graph}

Given a contig \(c\) in the set of contigs \(\Contigs{} = \Contigs{}_1 \cup \Contigs{}_2\) from the two assemblers (by \(\rev{c}\) we denote the reverse).
In the assembly-consensus graph \(ACG\), there exists a unique sequence \(p_c\) of vertices in \(V\) corresponding to the contig \(c\).

We respectively denote by \(\Links{}(c)\) and \(A(p_c)\) or \(A(c)\) the set of links in \Links{}, respectively of arcs in \(ACG\) representing contig \(c\).

For a contig \(c \in \Contigs{}\), \(\Fragments{}(c) \subset \Fragments{}\) is the set of all fragments of \(c\), while \(V(c)\) is the set of vertices in \(V\) representing \(c\).
