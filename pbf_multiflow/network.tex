\subsection{The network graph}\label{sec:pbmf:network}

All the MILP flow-based approaches are built on the top of a flow network, seen as a wrapper around the assembly graph \(G_{asm} = (V_\Contigs{}, A_\Links)\) (\zcref[S]{definition:assembly_graph}):

\begin{itemize}
  \item We add two vertices, \source{} and \sink{}, that are respectively the source and the sink vertices.
  \item The set \(\VSeeds{}\) contains the vertices associated with the seed contigs in \ContigSeeds{}.
  \item The source goes into all the vertices in \(V_{asm}\), and each of these vertices goes into the sink.
\end{itemize}

The main differences with the previous network graph in \zcref[S]{definition:pbf_iterbin:network_graph} are:
\begin{itemize}
  \item the source now connects all the vertices in \(V_{asm}\) and not only those of \(VSeeds{}\);
  \item there is no coverage score
\end{itemize}

\begin{definition}{Network graph}{pbmf:network_graph}
  Let \(N = (V, A, \source{}, \sink{}, \cov{\Contigs{}}, \plm{\Contigs{}})\) be the network graph, where:

  \begin{itemize}
    \item \( V = V_\Contigs{} \cup \Set*{\source{}, \sink{}} \) is the set of vertices with the source \(\source{}\) and the sink \(\sink{}\);
    \item \( A = A_\Links{} \cup \Set{(\source, v) \given v \in V_\Contigs{}} \cup \Set{(v, \sink) \given v \in V_\Contigs{}} \) is the set of link-arcs augmented with the source-arcs and the sink-arcs;
    \item \( \cov{\Contigs{}} \colon \Contigs{} \to \Reals{}_{> 0} \) is the coverage function;
    \item \( \plm{\Contigs{}} \colon \Contigs{} \to [-1, 1] \) is the plasmidness function.
  \end{itemize}
\end{definition}
