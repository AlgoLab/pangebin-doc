\begin{definition}{\MCF{} fine-tune variables}{milp:mcf_fine_tune_variables}
  \begin{itemize}
    \item \(x_c \in \Set{0, 1} \, \forall c \in \Contigs{}\) denoting whether the contig \(c\) is active or not.
    \item \(\ctggc{c}{b} \in \Set{0, 1} \, \forall (c, b) \in \Contigs{} \times K\) denoting whether the contig \(c\) participates in the solution and the plasmid GC content is in the interval \(b\) or not.
    \item \(\sgcbonus{j}{b} \in \Reals_{>=0} \, \forall \text{ share } j \in \Fragments{}, \forall b \in K\) corresponding to the best GC bonus over the GC content penalty for the share \(j\) when several of its contigs are active, given a GC content interval \(b\).
  \end{itemize}
\end{definition}

\begin{definition}{\MCF{} fine-tune constraint}{milp:mcf_fine_tune_constraints}
  A contig \(c \in \Contigs{}\), is active if and only if all the link-arcs defining \(c\) (or its reverse) are active:
  \begin{align}
    |A(c)|x_c & \leq \sum_{a \in A(c)} y_a \\
    |A(c^-)|x_c & \leq \sum_{a \in A(c^-)} y_a
  \end{align}

  \begin{notebox}
    The \enquote{if and only if} statement does not need to be completely written because of the objective function behaviour, see later.
  \end{notebox}

  For each contig \(c \in \Contigs{}\), for each GC content interval \(b \in K\), \(\ctggc{c}{b} = 1\) if and only if contig \(c\) is active and the \(b\) is the solution GC content interval:
  \begin{align}
    \ctggc{c}{b} & \leq x_c & \forall (c, b) \in \Contigs{} \times K \\
    \ctggc{c}{b} & \leq GC_b & \forall (c, b) \in \Contigs{} \times K \\
    \ctggc{c}{b} & \geq x_c + GC_b - 1 & \forall (c, b) \in \Contigs{} \times K
  \end{align}

  For each share \(j \in \Fragments{}\), for each GC content interval \(b \in K\), \(\sgcbonus{j}{b}\) is the best correction \(\dsgcpen{j}{c}{b}\) among the active contigs in \(\Contigs(j)\):
  \begin{equation}
    \sgcbonus{j}{b} \leq \sum_{\substack{
        d \in \Contigs(j) \\ \dsgcpen{j}{d}{b} \\ \geq \dsgcpen{j}{c}{b}
    }} \dsgcpen{j}{d}{b} \ctggc{d}{b} \quad
    \begin{array}[t]{@{}l@{}}
      \forall (j, c, b) \in \Fragments{} \times \Contigs{}(j) \times K \\
      j \text{ is a share}
    \end{array}
  \end{equation}

\end{definition}

\begin{definition}{\MCF{} objective function}{milp:mcf_objective}
  The objective function aims to:
  \begin{itemize}
    \item maximize the total flow \(F\)
    \item minimize the total GC content penalties, corrected by the bonuses:
      \begin{align}
        \begin{split}
          GCPenalties = & \sum_{\substack{
              i \in \Fragments{} \\
              b \in K
          }} \gcpen{i}{b} \fraggc{i}{b} \\
          & - \sum_{\substack{
              c \in \Contigs{} \\
              b \in K
          }} \dcgcpen{c}{b}\ctggc{c}{b} \\
          & - \sum_{\substack{
              \text{share } j \in \Fragments{} \\
              b \in K
          }} \sgcbonus{j}{b}
        \end{split}
      \end{align}
  \end{itemize}
  It results in the objective function:
  \begin{equation}
    \max ~ \parenth*{F - GCPenalties}
  \end{equation}
\end{definition}

\begin{todobox}
  Explain why we have added the \(GCPenalties\) in the \MCF{} objective function.

  Because this first approach is \emph{de novo}, we also use the GC content (also a de novo attribute) to extract plasmid candidate fragments.

  \begin{questionbox}
    It may be more precise to define this approach more as a de novo than a coarse-grain, as the \(GCPenalties\) is a fine-grain strategy over the coverage flow which is a coarse-grain one.
  \end{questionbox}
\end{todobox}
