\subsubsection{Correct the GC content score}\label{sec:method:mgc:fine_tuned}

The calculation of the GC probability score for a fragment \(i \in \Fragments{}\) does not rely to the belonging of \(i\) in its contigs \(c \in \Contigs{}(i)\).

\begin{questionbox}
  The plasmidness correction is simpler (see~\Cref{sec:method:mps:fine_tuned}) because in we just want to add a bonus.
  Here the question is: if the GC probability scores of the contigs have to be considered, can we choose one or a subset of contigs from those active to correct the GC content score?
  \begin{itemize}
    \item The contig with the highest absolute GC probability score difference?
    \item The contig with the highest GC Penalties probability score (bonus)?
    \item The contig with the lowest GC Penalties probability score (penalty)?
  \end{itemize}
  On related issue here is: should we choose only one difference, or a subset?
  The sum of relative GC probability score differences?

  \begin{todobox}
    Choose one and fix what follows
  \end{todobox}
\end{questionbox}

\begin{fixmebox}
  Because of the new attributes definitions, the corrections introductions and definitions must be adapted.
\end{fixmebox}

\begin{definition}{GC probability score bonus}{frag_gc_bonus}
  \begin{fixmebox}
    The GC penalty has became a GC prob.\ score: adapt the following definition.
  \end{fixmebox}
  Given a contig \(c \in \Contigs{}\) and a GC content interval \(b \in K\), we define \(\dcgcscore{c}{b}\) as the as a positive GC probability score bonus for the subcontigs of \(c\):
  \[
    \dcgcscore{c}{b} = \max \Set*{
      0,
      \sum_{\substack{
          i \in \Fragments{}(c) \\ i \text{ a subcontig}
      }}
      \parenth*{
        \frac{|i|}{|c|}\gcscore{c}{b} - \gcscore{i}{b}
      }
    }
  \]
  The bonus for the share depends on which contigs they belong.
  Given a share \(j \in \Fragments{}\) and a GC content interval \(b \in K\), we define \(\dsgcscore{j}{c}{b}\) as the positive difference between the GC penalty from the normalization of the fragment penalty according to the contig and the fragment penalty computed independently of the contigs it belongs:
  \[
    \dsgcscore{j}{c}{b} = \max \Set*{
      0,
      \frac{|j|}{|c|} \gcscore{c}{b} - \gcscore{j}{b}
    }
  \]
  \begin{questionbox}
    A real equivalence between a contig and the sequence of its fragments should imply the \enquote{bonus} to be possibly negative.
    How to argue in favour of a positive bonus rather than a real correction?

    Be aware that if the deltas can be negatives, then we must change the associated constraints
  \end{questionbox}
\end{definition}

\begin{ideabox}
  As for a contig \(c\),
  \[
    \gcscore{c}{b} \leq \dcgcscore{c}{b} + \sum_{\substack{
        j \in \Fragments{}(c) \\
        j \text{ is a share}
    }}\dsgcscore{j}{c}{b}
  \]
  If we don't want to count twice the share participations, we should define:
  \[
    \dcgcscore{c}{b} = \max \Set*{
      0,
      \gcscore{c}{b} - \sum_{i \in \Fragments{}(c)} \gcscore{i}{b}
    }
  \]
  and if a better contig is active, remove the share participation of the worst.

  \begin{questionbox}
    What should we not count twice a share participation, while if it is counted twice, it is because it is used several times (a repeat).

    Perhaps the best thing is just to add a bonus (or correct, negatively too)
  \end{questionbox}
\end{ideabox}

\begin{definition}{\MGC{} fine-tune variables}{mgc:milp:variables:fine_tune}
  \begin{itemize}
    \item \(x_c \in \Set{0, 1} \, \forall c \in \Contigs{}\) denoting whether the contig \(c\) is active or not.
    \item \(\ctggc{c}{b} \in \Set{0, 1} \, \forall (c, b) \in \Contigs{} \times K\) denoting whether the contig \(c\) participates in the solution and the plasmid GC content is in the interval \(b\) or not.
    \item \(\sgcbonus{j}{b} \in \Reals_{>=0} \, \forall \text{ share } j \in \Fragments{}, \forall b \in K\) corresponding to the best GC bonus over the GC content penalty for the share \(j\) when several of its contigs are active, given a GC content interval \(b\).
  \end{itemize}
\end{definition}

\begin{definition}{\MGC{} fine-tune constraint}{milp:mcf:constraints:fine_tune}
  A contig \(c \in \Contigs{}\), is active if and only if all the link-arcs defining \(c\) (or its reverse) are active:
  \begin{align}
    |A_\Links{}(c)|x_c & \leq \sum_{a \in A_\Links{}(c)} y_a \\
    |A_\Links{}(c^-)|x_c & \leq \sum_{a \in A_\Links{}(c^-)} y_a
  \end{align}

  \begin{questionbox}
    Depending of the question (bonus, penalty, relative correction), modelling the \enquote{if and only if} statement may be required.
  \end{questionbox}

  For each contig \(c \in \Contigs{}\), for each GC content interval \(b \in K\), \(\ctggc{c}{b} = 1\) if and only if contig \(c\) is active and the \(b\) is the solution GC content interval:
  \begin{align}
    \ctggc{c}{b} & \leq x_c & \forall (c, b) \in \Contigs{} \times K \\
    \ctggc{c}{b} & \leq GC_b & \forall (c, b) \in \Contigs{} \times K \\
    \ctggc{c}{b} & \geq x_c + GC_b - 1 & \forall (c, b) \in \Contigs{} \times K
  \end{align}

  For each share \(j \in \Fragments{}\), for each GC content interval \(b \in K\), \(\sgcbonus{j}{b}\) is the best correction \(\dsgcscore{j}{c}{b}\) among the active contigs in \(\Contigs(j)\):
  \begin{equation}
    \sgcbonus{j}{b} \leq \sum_{\substack{
        d \in \Contigs(j) \\ \dsgcscore{j}{d}{b} \\ \geq \dsgcscore{j}{c}{b}
    }} \dsgcscore{j}{d}{b} \ctggc{d}{b} \quad
    \begin{array}[t]{@{}l@{}}
      \forall (j, c, b) \in \Fragments{} \times \Contigs{}(j) \times K \\
      j \text{ is a share}
    \end{array}
  \end{equation}

\end{definition}

\begin{definition}{\MGC{} objective function with correction}{milp:mcf:objective:fine_tune}
  \begin{fixmebox}
    Depending of the question (bonus, penalty, relative correction), the objective function must be adapted.
  \end{fixmebox}
  The objective function aims to maximize the corrected GC probability score:
  \begin{equation}
    \max ~ GCCorrectedScore
  \end{equation}
  Where
  \begin{align}
    \begin{split}
      GCCorrectedScore = & \sum_{\substack{
          i \in \Fragments{} \\
          b \in K
      }} \gcprob{i}{b} \fraggc{i}{b} \\
      & - \sum_{\substack{
          c \in \Contigs{} \\
          b \in K
      }} \dcgcscore{c}{b}\ctggc{c}{b} \\
      & - \sum_{\substack{
          \text{share } j \in \Fragments{} \\
          b \in K
      }} \sgcbonus{j}{b}
    \end{split}
  \end{align}
\end{definition}
