\subsection{The Maximum Coverage-likelihood Flow problem \MCF{}}\label{sec:method:mcf}

The first binning stage consists in finding a flow maximizing the use of fragment coverage.
We refer to this subproblem as the \enquote{Maximum Coverage-likelihood Flow} problem (\MCF{}).
As a coarse-grain strategy, it consists in finding a connected component explaining the coverages, large enough to contain all the fragments of the in-building solution bin.
Indeed, passing through a loop or a cycle will not change the flow value. Only maximizing the flow can lead to use a minimum of fragments.
To overcome this bias, we score the use of the fragment coverages and maximize their total.
Bellow we describe the MILP model.

\begin{definition}{\MCF{} MILP variables}{milp:mcf_variables}
  \begin{itemize}
    \item \(x_v \in [0, 1] \, \forall v \in V \setminus \Set*{s, t}\) denoting whether the vertex \(v\) is active or not.
      The vertex corresponding to the fragment tail is active if, and only if the fragment in forward direction is active. Respectively, the head vertex is active if, and only if the reverse fragment is active. With the constraints it acts as a binary.
    \item \(y_a \in \Set{0, 1} \, \forall a \in A_\Links{}\) denoting whether the link-arc \(a\) is active or not.
    \item \(f_a \in \Reals_{\geq 0} \, \forall a \in A_\Links{}\) corresponding to the flow amount passing through the link-arc \(a\).
    \item \(F \in \Reals_{\geq 0}\) corresponding to the overall flow.
    \item \(F_a \in \Reals_{\geq 0} \, \forall a \in A_\Links{}\) playing the role of an intermediary variable to force the flow on each link-arc to be equal to the total flow.
    \item \(\alpha \in \Reals{}\) is the number of \emph{oriented} fragments in the solution connected graph, plus the source and the sink. With the constraints it acts as a positive integer.
    \item \(\beta_a \in \Reals_{\leq 0} \, \forall a \in A_\Links{}\) is strictly negative if the link-arc \(a\) participates in one of the solution subgraph's exploration tree.
      It acts as a negative integer, where the absolute value corresponds to the depth of the subtree defined by the successor.
  \end{itemize}
  Variables \(\alpha{}\) and the \(\beta_{A_\Links{}}\) serve to model the solution subgraph connectivity.
\end{definition}

\begin{definition}{\MCF{} MILP constraints}{milp:mcf_constraints}

  Exactly one link-arc outs of \(s\) is part of the solution (necessary to ensure the solution induced subgraph has only one connected component):
  \begin{todobox}
    Describe why it is not sufficient to have only one connected component
  \end{todobox}
  \begin{equation}
    \sum_{(s, v) \in A_\Links{}} y_{sv} = 1
  \end{equation}

  The flow through a link-arc \(a \in A_\Links{}\) is non-zero if \(a\) is active and cannot exceed its capacity:
  \begin{equation}
    \begin{split}
      f_{uv} \leq
      \begin{cases}
        \cov{j} y_{uv} & \text{if \(u = s\)} \\
        \cov{i} y_{uv} & \text{if \(v = t\)} \\
        \min\Set*{\cov{i}, \cov{j}} y_{uv} & \text{otherwise}
      \end{cases} \quad
      \begin{split}
        \forall (u, v) \in A_\Links{} \\
        i = vfrag(u) \\
        j = vfrag(v)
      \end{split}
    \end{split}
  \end{equation}

  The cumulative flow through a fragment \(i \in \Fragments{}\) cannot exceed its read coverage:
  \begin{equation}
    inflow(i) \leq \cov{i} \quad \forall i \in \Fragments{}
  \end{equation}
  Where \(\displaystyle inflow(i) = \sum_{(u, i_t) \in A_\Links{}} f_{ui_t} + \sum_{(u, i_h) \in A_\Links{}} f_{ui_h}\) where \(i_t\) and \(i_h\) respectively stand for the tail and the head vertices of fragment \(i\).

  The cumulative flow into a fragment \(i\) should be equal to the cumulative flow out of it. Same for the reverse \(i^-\):
  \begin{align}
    \sum_{(u, i_t) \in A_\Links{}} f_{u i_t} & = \sum_{(i_h, w) \in A_\Links{}} f_{i_h w} & \forall i \in \Fragments{} \\
    \sum_{(u, i_h) \in A_\Links{}} f_{u i_h} & = \sum_{(i_t, w) \in A_\Links{}} f_{i_t w} & \forall i \in \Fragments{}
  \end{align}

  The total flow value \(F\) equals to the flow out of \(s\) and into \(t\):
  \begin{align}
    F & = \sum_{(s, v) \in A_\Links{}} f_{sv} \\
    F & = \sum_{(v, t) \in A_\Links{}} f_{vt}
  \end{align}

  The fragments involved in an active link-arc must also be active:
  \begin{equation}
    y_{uv} \leq
    \begin{cases}
      x_v & \text{if \(u = s\)} \\
      x_u & \text{if \(v = t\)} \\
      \min\Set{x_u, x_v} & \text{otherwise}
    \end{cases} \quad \forall (u, v) \in A_\Links{}
  \end{equation}

  An active vertex implies at least one active link-arc incoming to it:
  \begin{questionbox}
    Are these constraints necessary?
  \end{questionbox}
  \begin{equation}
    x_v \leq \sum_{(u, v) \in A_\Links{}} y_{u v} \quad \forall v \in V \setminus \Set{s, t}
  \end{equation}

  A positive flow implies the two incident vertices are in the component:
  \begin{equation}
    1 - y_{uv} \geq
    \begin{cases}
      1 - x_v & \text{if \(u = s\)} \\
      x_u - 1 & \text{if \(v = t\)} \\
      x_u - x_v & \text{otherwise}
    \end{cases}
    \quad \forall (u, v) \in A_\Links{}
  \end{equation}

  The cumulative depth of the subtree in the exploration tree from the source equals at least the number of active vertices minus 1.
  \begin{align}
    \alpha + \sum_{(s, w) \in A_\Links{}} \beta_{sw} \leq 1
  \end{align}

  Two incident arcs in the exploration tree are distanced by one.
  \begin{align}
    \sum_{(v, w) \in A_\Links{}} \beta_{vw} - \sum_{(u, v) \in A_\Links{}} \beta_{uv} \leq 1 \quad \forall v \in V \setminus \Set*{s}
  \end{align}

  An arc participates in the exploration tree implies the arc is active.
  \begin{align}
    \beta_a \geq - y_a |V| \quad \forall a \in A_\Links{}
  \end{align}

  The number of active vertices equals \(\alpha{}\) (SAT constraint):
  \begin{align}
    2 + \sum_{v \in V \setminus \Set*{s, t}} x_v = \alpha
  \end{align}

  An active link-arc has a flow at least \(F\).
  \begin{align}
    F_a & \geq F - (1 - y_a) \max_{i \in \SeedFrags{}}\Set{\cov{i}} & \forall a \in A_\Links{} \\
    F_a & \leq F & \forall a \in A_\Links{} \\
    F_a & \leq f_a & \forall a \in A_\Links{} \label{const:arc_flow_at_least_total_flow}
  \end{align}
  \begin{infobox}
    The above constraints minimize the number of out link-arcs for each fragment.
  \end{infobox}
  \begin{missingproofbox}
    If several link-arcs out of a given fragment,
    then this fragment is repeated (immediate).
  \end{missingproofbox}
  \begin{questionbox}
    What is the meaning for a fragment to have a cumulative flow that is not a multiple of \(F\)?
    By keeping the flow real, can we smartly force the cumulative flow to be a multiple of \(F\)?

    \begin{notebox}
      Because the sequencing coverage does not represent an integer number of copy of the sequenced genome.
    \end{notebox}
  \end{questionbox}
\end{definition}

\begin{definition}{\MCF{} MILP objective function}{milp:mcf_objective}
  \begin{newfeatbox}
    Here we maximize the coverage scores, not only the coverage flow.
  \end{newfeatbox}
  The objective function aims to maximize the total coverage scores:
  \begin{equation}
    \max ~ CoverageScore
  \end{equation}
  where \(
    \displaystyle CoverageScore = %
    \sum_{i \in \Fragments{}} \parenth*{2 \frac{inflow(i)}{\cov{i}} - 1}
  \).
\end{definition}