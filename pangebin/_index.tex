\section{Pangebin method}

\begin{todobox}
  Summarize thanks to an algorithm.
\end{todobox}

\begin{todobox}
  Generalize the idea of coarse-to-fine-grain approach by splitting all the objectives.

  \begin{enumerate}[label=\roman*.]
    \item Find max flow
    \item With flow near to max flow, find best GC score
    \item With flow near to max flow, with GC score near to best GC score, find best plasmidness score
    \item \textbf{[Future]} With blabla, find best plasmid-type subgraph
  \end{enumerate}
\end{todobox}

Pangebin is based on the PlasBin-flow approach~\cite{manePlasBinflowFlowbasedMILP2023} and adapts its method to be applied to pan-assembly graphs.

\begin{newfeatbox}
  At the opposite of PlasBin-flow, we split each objective expression into different mono-objective functions that define different optimization problems.
\end{newfeatbox}

However, Pangebin tackles the binning of the fragments with a coarse-to-fine-grain approach in three stages (each one refering to an optimization problem) iteratively applied on smaller pan-assembly graph:

\begin{enumerate}[label=\roman*.]
  \item Find the maximum coverage flow (\MCF{} problem, \Cref{sec:method:mcf})
  \item Find the maximum GC probability score (\MGC{} problem)
  \item Find the maximum plasmidness score (\MPS{} problem)
\end{enumerate}

We model these three subproblems as dedicated flow formulations in the panassembly network (\Cref{definition:panassembly_network}) with Mixed Integer Linear Programming (MILP):

\begin{itemize}
  \item To keep the direction of the flow along the link-edges in \(E_\Links{}\), we transform each link-edge to two link-arcs with the function \(etoa \colon \Set*{u, v} \mapsto \Set*{(u, v), (v, u)}\).
  \item We add two vertices, \(s\) and \(t\) that are respectively the source and the sink vertices.
  \item The set \(\VSeed{}\) contains the vertices associated with the seed fragments.
    The source goes into each vertex in \(\VSeed{}\), and each vertex in \(V_t \cup V_h\) goes into the sink.
\end{itemize}

\begin{definition}{Pan-assembly network}{panassembly_network}
  Let \(N = (V, A_\Fragments{} \cup A_\Links{}, s, t, \cov{\Fragments{}}, \gcscore{\Fragments{}}{K}, \plm{\Fragments{}})\) be the pan-assembly network, where:

  \begin{itemize}
    \item \(V = V_t \cup V_h \cup \Set*{s, t}\) is the set of vertices with the source \(s\) and the sink \(t\);
    \item \(A_\Fragments{} = \bigcup_{e \in E_\Fragments{}} etoa(e)\) is the set of fragment-arcs;
    \item \(A_\Links{} = \bigcup_{e \in E_\Links{}} etoa(e) \cup \Set{(s, v) \given v \in \VSeed{}} \cup \Set{(v, t) \given v \in V}\) is the set of link-arcs augmented with the seed-arcs and the sink-arcs;
    \item \(\cov{\Fragments{}} \colon \Fragments{} \to \Reals{}_{> 0}\) is the coverage function;
    \item \(\gcscore{\Fragments{}}{K} \colon \Fragments{} \times K \to [-1, 1]\) is the GC score function;
    \item \(\plm{\Fragments{}} \colon \Fragments{} \to [0, 1]\) is the plasmidness function.
  \end{itemize}
\end{definition}

In what follow, we define either a fragment, a contig or an arc to be \enquote{active} if it participates in the solution, i.e.\ the flow passes through it.

\begin{newfeatbox}
  We model a connected-flow directly in the MILP model.

  \begin{notebox}
    The variables \(x\) are now defined in the vertex set rather than the fragment set.
  \end{notebox}
\end{newfeatbox}

Furthermore, Pangebin answers one of the PlasBin-flow discussion point: we ensure the induced solution subgraph defined by arcs with positive flow is connected.

\begin{todobox}
  Describe the update of the graph between the iterations (for the two problems).

  \begin{enumerate}
    \item reduce the coverage
    \item remove from the contig set the contigs for which one of its fragment has a null coverage
  \end{enumerate}
\end{todobox}

\subsection{The Maximum Coverage-likelihood Flow problem \MCF{}}\label{sec:method:mcf}

The first binning stage consists in finding a flow maximizing the use of fragment coverage.
We refer to this subproblem as the \enquote{Maximum Coverage-likelihood Flow} problem (\MCF{}).
As a coarse-grain strategy, it consists in finding a connected component explaining the coverages, large enough to contain all the fragments of the in-building solution bin.
Indeed, passing through a loop or a cycle will not change the flow value. Only maximizing the flow can lead to use a minimum of fragments.
To overcome this bias, we score the use of the fragment coverages and maximize their total.
Bellow we describe the MILP model.

\begin{definition}{\MCF{} MILP variables}{milp:mcf_variables}
  \begin{itemize}
    \item \(x_v \in [0, 1] \, \forall v \in V \setminus \Set*{s, t}\) denoting whether the vertex \(v\) is active or not.
      The vertex corresponding to the fragment tail is active if, and only if the fragment in forward direction is active. Respectively, the head vertex is active if, and only if the reverse fragment is active. With the constraints it acts as a binary.
    \item \(y_a \in \Set{0, 1} \, \forall a \in A_\Links{}\) denoting whether the link-arc \(a\) is active or not.
    \item \(f_a \in \Reals_{\geq 0} \, \forall a \in A_\Links{}\) corresponding to the flow amount passing through the link-arc \(a\).
    \item \(F \in \Reals_{\geq 0}\) corresponding to the overall flow.
    \item \(F_a \in \Reals_{\geq 0} \, \forall a \in A_\Links{}\) playing the role of an intermediary variable to force the flow on each link-arc to be equal to the total flow.
    \item \(\alpha \in \Reals{}\) is the number of \emph{oriented} fragments in the solution connected graph, plus the source and the sink. With the constraints it acts as a positive integer.
    \item \(\beta_a \in \Reals_{\leq 0} \, \forall a \in A_\Links{}\) is strictly negative if the link-arc \(a\) participates in one of the solution subgraph's exploration tree.
      It acts as a negative integer, where the absolute value corresponds to the depth of the subtree defined by the successor.
  \end{itemize}
  Variables \(\alpha{}\) and the \(\beta_{A_\Links{}}\) serve to model the solution subgraph connectivity.
\end{definition}

\begin{definition}{\MCF{} MILP constraints}{milp:mcf_constraints}

  Exactly one link-arc outs of \(s\) is part of the solution (necessary to ensure the solution induced subgraph has only one connected component):
  \begin{todobox}
    Describe why it is not sufficient to have only one connected component
  \end{todobox}
  \begin{equation}
    \sum_{(s, v) \in A_\Links{}} y_{sv} = 1
  \end{equation}

  The flow through a link-arc \(a \in A_\Links{}\) is non-zero if \(a\) is active and cannot exceed its capacity:
  \begin{equation}
    \begin{split}
      f_{uv} \leq
      \begin{cases}
        \cov{j} y_{uv} & \text{if \(u = s\)} \\
        \cov{i} y_{uv} & \text{if \(v = t\)} \\
        \min\Set*{\cov{i}, \cov{j}} y_{uv} & \text{otherwise}
      \end{cases} \quad
      \begin{split}
        \forall (u, v) \in A_\Links{} \\
        i = vfrag(u) \\
        j = vfrag(v)
      \end{split}
    \end{split}
  \end{equation}

  The cumulative flow through a fragment \(i \in \Fragments{}\) cannot exceed its read coverage:
  \begin{equation}
    inflow(i) \leq \cov{i} \quad \forall i \in \Fragments{}
  \end{equation}
  Where \(\displaystyle inflow(i) = \sum_{(u, i_t) \in A_\Links{}} f_{ui_t} + \sum_{(u, i_h) \in A_\Links{}} f_{ui_h}\) where \(i_t\) and \(i_h\) respectively stand for the tail and the head vertices of fragment \(i\).

  The cumulative flow into a fragment \(i\) should be equal to the cumulative flow out of it. Same for the reverse \(i^-\):
  \begin{align}
    \sum_{(u, i_t) \in A_\Links{}} f_{u i_t} & = \sum_{(i_h, w) \in A_\Links{}} f_{i_h w} & \forall i \in \Fragments{} \\
    \sum_{(u, i_h) \in A_\Links{}} f_{u i_h} & = \sum_{(i_t, w) \in A_\Links{}} f_{i_t w} & \forall i \in \Fragments{}
  \end{align}

  The total flow value \(F\) equals to the flow out of \(s\) and into \(t\):
  \begin{align}
    F & = \sum_{(s, v) \in A_\Links{}} f_{sv} \\
    F & = \sum_{(v, t) \in A_\Links{}} f_{vt}
  \end{align}

  The fragments involved in an active link-arc must also be active:
  \begin{equation}
    y_{uv} \leq
    \begin{cases}
      x_v & \text{if \(u = s\)} \\
      x_u & \text{if \(v = t\)} \\
      \min\Set{x_u, x_v} & \text{otherwise}
    \end{cases} \quad \forall (u, v) \in A_\Links{}
  \end{equation}

  An active vertex implies at least one active link-arc incoming to it:
  \begin{questionbox}
    Are these constraints necessary?
  \end{questionbox}
  \begin{equation}
    x_v \leq \sum_{(u, v) \in A_\Links{}} y_{u v} \quad \forall v \in V \setminus \Set{s, t}
  \end{equation}

  A positive flow implies the two incident vertices are in the component:
  \begin{equation}
    1 - y_{uv} \geq
    \begin{cases}
      1 - x_v & \text{if \(u = s\)} \\
      x_u - 1 & \text{if \(v = t\)} \\
      x_u - x_v & \text{otherwise}
    \end{cases}
    \quad \forall (u, v) \in A_\Links{}
  \end{equation}

  The cumulative depth of the subtree in the exploration tree from the source equals at least the number of active vertices minus 1.
  \begin{align}
    \alpha + \sum_{(s, w) \in A_\Links{}} \beta_{sw} \leq 1
  \end{align}

  Two incident arcs in the exploration tree are distanced by one.
  \begin{align}
    \sum_{(v, w) \in A_\Links{}} \beta_{vw} - \sum_{(u, v) \in A_\Links{}} \beta_{uv} \leq 1 \quad \forall v \in V \setminus \Set*{s}
  \end{align}

  An arc participates in the exploration tree implies the arc is active.
  \begin{align}
    \beta_a \geq - y_a |V| \quad \forall a \in A_\Links{}
  \end{align}

  The number of active vertices equals \(\alpha{}\) (SAT constraint):
  \begin{align}
    2 + \sum_{v \in V \setminus \Set*{s, t}} x_v = \alpha
  \end{align}

  An active link-arc has a flow at least \(F\).
  \begin{align}
    F_a & \geq F - (1 - y_a) \max_{i \in \SeedFrags{}}\Set{\cov{i}} & \forall a \in A_\Links{} \\
    F_a & \leq F & \forall a \in A_\Links{} \\
    F_a & \leq f_a & \forall a \in A_\Links{} \label{const:arc_flow_at_least_total_flow}
  \end{align}
  \begin{infobox}
    These constraints do not force the arc flows to be a multiple of \(F\), see~\Cref{proposition:inflow_is_not_multiple_of_total_flow}.
  \end{infobox}
  \begin{infobox}
    The above constraints minimize the number of out link-arcs for each fragment.
  \end{infobox}
  \begin{missingproofbox}
    If several link-arcs out of a given fragment,
    then this fragment is repeated (immediate).
  \end{missingproofbox}
  \begin{questionbox}
    What is the meaning for a fragment to have a cumulative flow that is not a multiple of \(F\)?
    By keeping the flow real, can we smartly force the cumulative flow to be a multiple of \(F\)?

    \begin{notebox}
      Because the sequencing coverage does not represent an integer number of copy of the sequenced genome.
    \end{notebox}
  \end{questionbox}
\end{definition}

\begin{definition}{\MCF{} MILP objective function}{milp:mcf_objective}
  \begin{newfeatbox}
    Here we maximize the coverage scores, not only the coverage flow.
  \end{newfeatbox}
  The objective function aims to maximize the total coverage scores:
  \begin{equation}
    \max ~ CoverageScore
  \end{equation}
  where \(
    \displaystyle CoverageScore = %
    \sum_{i \in \Fragments{}} \parenth*{2 \frac{inflow(i)}{\cov{i}} - 1}
  \).
\end{definition}

\subsection{The Maximum GC Probability Score problem \MGC{}}\label{meth:max_gc_score}

The second binning stage consists in finding a flow maximizing the GC probability scores, by fixing the coverage score flow.
We refer to this subproblem as the \enquote{Maximum GC Probability Score} problem (\MGC{}).
In the following we describe the MILP model.

\begin{definition}{\MGC{} MILP variables}{milp:mgc_variables}
    We complete \Cref{definition:milp:mcf_variables} with the following variables:
    \begin{itemize}
        \item \(GC_b \in \Set{0, 1} \, \forall b \in K\) denoting whether the plasmid GC content is in the GC content interval \(b\) or not.
        \item \(\fraggc{i}{b} \in \Set{0, 1} \, \forall (i, b) \in \Fragments{} \times K\) denoting whether the fragment \(i\) is active and the plasmid GC content is in the interval \(b\) or not.
    \end{itemize}
\end{definition}


\begin{definition}{\MGC{} MILP constraints}{milp:mgc_constraints}
    To the constraints in \Cref{definition:milp:mcf_constraints} we add the following:

    The coverage score must be near to the optimal coverage score \(\Phi\) found in \MCF{}:
    \begin{equation}
        \gamma_1 \Phi \leq  CoverageScore \quad \gamma_1 \in (0.5, 1]
    \end{equation}

    The plasmid GC content is in exactly one GC content interval \(b \in K\):
    \begin{equation}
        \sum_{b \in K} GC_b = 1 
    \end{equation}

    For each fragment \(i \in \Fragments{}\), for each GC content interval \(b \in K\), \(\fraggc{i}{b} = 1\) if and only if fragment \(i\) is active and the \(b\) is the solution GC content interval:
    \begin{align}
        \fraggc{i}{b} & \leq x_i & \forall (i, b) \in \Fragments{} \times K \\
        \fraggc{i}{b} & \leq GC_b & \forall (i, b) \in \Fragments{} \times K \\
        \fraggc{i}{b} & \geq x_i + GC_b - 1 & \forall (i, b) \in \Fragments{} \times K
    \end{align}
\end{definition}

\begin{definition}{\MGC{} MILP objective function}{milp:mgc_objective}
    The objective function aims to maximize the GC probability score:
    \begin{equation}
        \max ~ GCProbabilityScore
    \end{equation}
    where \(
        \displaystyle GCProbabilityScore = \sum_{\substack{
            i \in \Fragments{} \\
            b \in K
        }} \gcscore{i}{b} \fraggc{i}{b}%
    \).
\end{definition}

\subsection{The Maximum Plasmidness Score problem \MPS{}}\label{sec:method:mps}

The third binning stage consists in finding a flow maximizing the plasmidness scores, by fixing both the coverage flow and the GC probability score.
We refer to this subproblem as the \enquote{Maximum Plasmidness Score} problem (\MPS{}).
In the following we describe the MILP model.

\begin{definition}{\MPS{} MILP variables}{milp:mps_variables}
  We complete \Cref{definition:milp:mgc_variables} with the following variables:
  \(\frag{i} \in [0, 1] \, \forall i \in \Fragments{}\) denoting whether fragment \(i\) is active or not. With the constraint it acts as a binary.
\end{definition}

\begin{definition}{\MGC{} MILP constraints}{milp:mps_constraints}
  To the constraints in \Cref{definition:milp:mgc_constraints} we add the following:

  The total GC probability score value must be near to the total GC probability score value \(\Psi{}\) found in \MGC{}:
  \begin{equation}
    \gamma_2 \Psi \leq GCProbabilityScore \quad \gamma_2 \in (0.5, 1]
  \end{equation}

  The fragment is active if at least one of its extremities is active:
  \begin{align}
    \frag{i} \leq x_{i_t} + x_{i_h} \quad \forall i \in \Fragments{} \\
    x_{i_t} \leq \frag{i} \quad \forall i \in \Fragments{} \\
    x_{i_h} \leq \frag{i} \quad \forall i \in \Fragments{}
  \end{align}
\end{definition}

\begin{definition}{\MGC{} MILP objective function}{milp:mps_objective}
  \begin{newfeatbox}
    Here the plasmidness either act as a bonus if they are more than 0.5, or a penalty if they are less than 0.5.
  \end{newfeatbox}
  The objective function aims to maximize the plasmidness scores:
  \begin{equation}
    \max ~ PlasmidnessScore
  \end{equation}
  where \(\displaystyle PlasmidnessScore = \sum_{i \in \Fragments{}} 2 (\plm{i} - 0.5)\frag{i}\).
\end{definition}

\subsection{Fine-tuning the MILP formulations}

\begin{warningbox}
  This part is in development
\end{warningbox}

\begin{fixmebox}
  Do not use alpha as an example, because it is already a MILP variable.
\end{fixmebox}
Because of the fragmentation, for a given attribute, we do not have a strict attribute-equivalence between each contig and its corresponding fragment path in the pan-assembly graph.
Formally, if \(\alpha{}\) is an attribute defined either for the contigs and the fragments,
\(\exists c \in \Contigs{} \text{ s.t. } \alpha_c \neq \sum_{i \in \Fragments{}(c)} \alpha_i\).
Especially when \(\alpha_c\) is inferior than the sum, if the flow passes through the fragment path corresponding to this contig, we would like to positively correct the sum to obtain \(\alpha_c\).

In what follows, we associate new attributes to the fragments corresponding to the possible positive correction we may add to the objective values.
We then adapt the MILP formulations for the corrections.

\subsubsection{Fragment correction attributes}

\begin{fixmebox}
    Because of the new attributes definitions, the corrections introductions and definitions must be adapted.
\end{fixmebox}

\begin{definition}{GC probability score bonus}{frag_gc_bonus}
    \begin{fixmebox}
        The GC penalty become GC prob. score: adapt the following definition.
    \end{fixmebox}
    Given a contig \(c \in \Contigs{}\) and a GC content interval \(b \in K\), we define \(\dcgcscore{c}{b}\) as the as a positive GC probability score bonus for the subcontigs of \(c\):
    \[
        \dcgcscore{c}{b} = \max \Set*{
            0, 
            \sum_{\substack{
                i \in \Fragments{}(c) \\ i \text{ a subcontig}
            }} 
            \parenth*{
                \frac{|i|}{|c|}\gcscore{c}{b} - \gcscore{i}{b}
            }
        } 
    \]
    The bonus for the share depends on which contigs they belong.
    Given a share \(j \in \Fragments{}\) and a GC content interval \(b \in K\), we define \(\dsgcscore{j}{c}{b}\) as the positive difference between the GC penalty from the normalization of the fragment penalty according to the contig and the fragment penalty computed independently of the contigs it belongs:
    \[
        \dsgcscore{j}{c}{b} = \max \Set*{
            0,
            \frac{|j|}{|c|} \gcscore{c}{b} - \gcscore{j}{b}
        }
    \]
    \begin{questionbox}
        A real equivalence between a contig and the sequence of its fragments should imply the \enquote{bonus} to be possibly negative.
        How to argue in favour of a positive bonus rather than a real correction?

        Be aware that if the deltas can be negatives, then we must change the associated constraints
    \end{questionbox}
\end{definition}

\begin{ideabox}
    As for a contig \(c\), 
    \[
    \gcscore{c}{b} \leq \dcgcscore{c}{b} + \sum_{\substack{
        j \in \Fragments{}(c) \\
        j \text{ is a share}
    }}\dsgcscore{j}{c}{b}
    \]
    If we don't want to count twice the share participations, we should define:
    \[
    \dcgcscore{c}{b} = \max \Set*{
            0, 
            \gcscore{c}{b} - \sum_{i \in \Fragments{}(c)} \gcscore{i}{b}
        } 
    \]
    and if a better contig is active, remove the share participation of the worst.

    \begin{questionbox}
        What should we not count twice a share participation, while if it is counted twice, it is because it is used several times (a repeat).

        Perhaps the best thing is just to add a bonus (or correct, negatively too)
    \end{questionbox}
\end{ideabox}

\begin{notebox}
    The constraints were written but as the discussions are changing a lot, I don't put them in the doc now (see \url{pangebin/fine_tune/to_sort.tex} file).  
\end{notebox}
% \subsection{The Maximum Coverage-likelihood Flow problem \MCF{}}\label{sec:method:mcf}

The first binning stage consists in finding a flow maximizing the use of fragment coverage.
We refer to this subproblem as the \enquote{Maximum Coverage-likelihood Flow} problem (\MCF{}).
As a coarse-grain strategy, it consists in finding a connected component explaining the coverages, large enough to contain all the fragments of the in-building solution bin.
Indeed, passing through a loop or a cycle will not change the flow value. Only maximizing the flow can lead to use a minimum of fragments.
To overcome this bias, we score the use of the fragment coverages and maximize their total.
Bellow we describe the MILP model.

\begin{definition}{\MCF{} MILP variables}{milp:mcf_variables}
  \begin{itemize}
    \item \(x_v \in [0, 1] \, \forall v \in V \setminus \Set*{s, t}\) denoting whether the vertex \(v\) is active or not.
      The vertex corresponding to the fragment tail is active if, and only if the fragment in forward direction is active. Respectively, the head vertex is active if, and only if the reverse fragment is active. With the constraints it acts as a binary.
    \item \(y_a \in \Set{0, 1} \, \forall a \in A_\Links{}\) denoting whether the link-arc \(a\) is active or not.
    \item \(f_a \in \Reals_{\geq 0} \, \forall a \in A_\Links{}\) corresponding to the flow amount passing through the link-arc \(a\).
    \item \(F \in \Reals_{\geq 0}\) corresponding to the overall flow.
    \item \(F_a \in \Reals_{\geq 0} \, \forall a \in A_\Links{}\) playing the role of an intermediary variable to force the flow on each link-arc to be equal to the total flow.
    \item \(\alpha \in \Reals{}\) is the number of \emph{oriented} fragments in the solution connected graph, plus the source and the sink. With the constraints it acts as a positive integer.
    \item \(\beta_a \in \Reals_{\leq 0} \, \forall a \in A_\Links{}\) is strictly negative if the link-arc \(a\) participates in one of the solution subgraph's exploration tree.
      It acts as a negative integer, where the absolute value corresponds to the depth of the subtree defined by the successor.
  \end{itemize}
  Variables \(\alpha{}\) and the \(\beta_{A_\Links{}}\) serve to model the solution subgraph connectivity.
\end{definition}

\begin{definition}{\MCF{} MILP constraints}{milp:mcf_constraints}

  Exactly one link-arc outs of \(s\) is part of the solution (necessary to ensure the solution induced subgraph has only one connected component):
  \begin{todobox}
    Describe why it is not sufficient to have only one connected component
  \end{todobox}
  \begin{equation}
    \sum_{(s, v) \in A_\Links{}} y_{sv} = 1
  \end{equation}

  The flow through a link-arc \(a \in A_\Links{}\) is non-zero if \(a\) is active and cannot exceed its capacity:
  \begin{equation}
    \begin{split}
      f_{uv} \leq
      \begin{cases}
        \cov{j} y_{uv} & \text{if \(u = s\)} \\
        \cov{i} y_{uv} & \text{if \(v = t\)} \\
        \min\Set*{\cov{i}, \cov{j}} y_{uv} & \text{otherwise}
      \end{cases} \quad
      \begin{split}
        \forall (u, v) \in A_\Links{} \\
        i = vfrag(u) \\
        j = vfrag(v)
      \end{split}
    \end{split}
  \end{equation}

  The cumulative flow through a fragment \(i \in \Fragments{}\) cannot exceed its read coverage:
  \begin{equation}
    inflow(i) \leq \cov{i} \quad \forall i \in \Fragments{}
  \end{equation}
  Where \(\displaystyle inflow(i) = \sum_{(u, i_t) \in A_\Links{}} f_{ui_t} + \sum_{(u, i_h) \in A_\Links{}} f_{ui_h}\) where \(i_t\) and \(i_h\) respectively stand for the tail and the head vertices of fragment \(i\).

  The cumulative flow into a fragment \(i\) should be equal to the cumulative flow out of it. Same for the reverse \(i^-\):
  \begin{align}
    \sum_{(u, i_t) \in A_\Links{}} f_{u i_t} & = \sum_{(i_h, w) \in A_\Links{}} f_{i_h w} & \forall i \in \Fragments{} \\
    \sum_{(u, i_h) \in A_\Links{}} f_{u i_h} & = \sum_{(i_t, w) \in A_\Links{}} f_{i_t w} & \forall i \in \Fragments{}
  \end{align}

  The total flow value \(F\) equals to the flow out of \(s\) and into \(t\):
  \begin{align}
    F & = \sum_{(s, v) \in A_\Links{}} f_{sv} \\
    F & = \sum_{(v, t) \in A_\Links{}} f_{vt}
  \end{align}

  The fragments involved in an active link-arc must also be active:
  \begin{equation}
    y_{uv} \leq
    \begin{cases}
      x_v & \text{if \(u = s\)} \\
      x_u & \text{if \(v = t\)} \\
      \min\Set{x_u, x_v} & \text{otherwise}
    \end{cases} \quad \forall (u, v) \in A_\Links{}
  \end{equation}

  An active vertex implies at least one active link-arc incoming to it:
  \begin{questionbox}
    Are these constraints necessary?
  \end{questionbox}
  \begin{equation}
    x_v \leq \sum_{(u, v) \in A_\Links{}} y_{u v} \quad \forall v \in V \setminus \Set{s, t}
  \end{equation}

  A positive flow implies the two incident vertices are in the component:
  \begin{equation}
    1 - y_{uv} \geq
    \begin{cases}
      1 - x_v & \text{if \(u = s\)} \\
      x_u - 1 & \text{if \(v = t\)} \\
      x_u - x_v & \text{otherwise}
    \end{cases}
    \quad \forall (u, v) \in A_\Links{}
  \end{equation}

  The cumulative depth of the subtree in the exploration tree from the source equals at least the number of active vertices minus 1.
  \begin{align}
    \alpha + \sum_{(s, w) \in A_\Links{}} \beta_{sw} \leq 1
  \end{align}

  Two incident arcs in the exploration tree are distanced by one.
  \begin{align}
    \sum_{(v, w) \in A_\Links{}} \beta_{vw} - \sum_{(u, v) \in A_\Links{}} \beta_{uv} \leq 1 \quad \forall v \in V \setminus \Set*{s}
  \end{align}

  An arc participates in the exploration tree implies the arc is active.
  \begin{align}
    \beta_a \geq - y_a |V| \quad \forall a \in A_\Links{}
  \end{align}

  The number of active vertices equals \(\alpha{}\) (SAT constraint):
  \begin{align}
    2 + \sum_{v \in V \setminus \Set*{s, t}} x_v = \alpha
  \end{align}

  An active link-arc has a flow at least \(F\).
  \begin{align}
    F_a & \geq F - (1 - y_a) \max_{i \in \SeedFrags{}}\Set{\cov{i}} & \forall a \in A_\Links{} \\
    F_a & \leq F & \forall a \in A_\Links{} \\
    F_a & \leq f_a & \forall a \in A_\Links{} \label{const:arc_flow_at_least_total_flow}
  \end{align}
  \begin{infobox}
    These constraints do not force the arc flows to be a multiple of \(F\), see~\Cref{proposition:inflow_is_not_multiple_of_total_flow}.
  \end{infobox}
  \begin{infobox}
    The above constraints minimize the number of out link-arcs for each fragment.
  \end{infobox}
  \begin{missingproofbox}
    If several link-arcs out of a given fragment,
    then this fragment is repeated (immediate).
  \end{missingproofbox}
  \begin{questionbox}
    What is the meaning for a fragment to have a cumulative flow that is not a multiple of \(F\)?
    By keeping the flow real, can we smartly force the cumulative flow to be a multiple of \(F\)?

    \begin{notebox}
      Because the sequencing coverage does not represent an integer number of copy of the sequenced genome.
    \end{notebox}
  \end{questionbox}
\end{definition}

\begin{definition}{\MCF{} MILP objective function}{milp:mcf_objective}
  \begin{newfeatbox}
    Here we maximize the coverage scores, not only the coverage flow.
  \end{newfeatbox}
  The objective function aims to maximize the total coverage scores:
  \begin{equation}
    \max ~ CoverageScore
  \end{equation}
  where \(
    \displaystyle CoverageScore = %
    \sum_{i \in \Fragments{}} \parenth*{2 \frac{inflow(i)}{\cov{i}} - 1}
  \).
\end{definition}

% \subsection{Plasmid-properties guided flow}\label{meth:plasmid_properties_guided_flow}

The second stage consists is a reference fine-grain approach to extract plasmid fragments in the pan-assembly subgraph induced by the solution given by \MCF{}.
We refer to this problem by the \enquote{Plasmid-Properties Guided Flow} problem (\PPGF{}).

\begin{todobox}
    Explain GC content interval set \(K\).
\end{todobox}

\begin{todobox}
    Finding best plasmid score flow under max flow constraint
\end{todobox}

\begin{definition}{\PPGF{} MILP variables}{milp:ppgf_vars}
    In addition to the variables in \Cref{definition:milp:common_variables}, we define the following decision variables:
    \begin{itemize}
        \item \(\plmbonus{j} \in \mathbb{R}_{\geq 0} \, \forall j \in \Fragments{} \mid j \text{ is a share}\) corresponding to the best correction \(\dplm{c}{j}\) among the contigs in \(\ContigSet(j)\) participating in the solution.
    \end{itemize}
\end{definition}

\begin{definition}{\PPGF{} MILP constraints}{milp:ppgf_constraints}
    We add the next constraints to the constraints in \Cref{definition:milp:common_constraints}.

    The total flow value must be near to the total flow value \(\Phi\) found in \MCF{}:
    \begin{equation}
        \gamma \Phi \leq  F \quad \gamma \in (0.5, 1]
    \end{equation}

    The total GC content penalty should not be so much worst than the total GC content penalty \(\Psi\) found in \MCF{}:
    \begin{equation}
       GCPenalties \leq (1 + \epsilon) \Psi \quad \epsilon \in [0, 0.5)
    \end{equation}

    \begin{refactorbox}
        \begin{fixmebox}
            Argument order
        \end{fixmebox}
        For each share \(j \in \Fragments{}\), \(\plmbonus{j}\) is the best correction \(\dplm{c}{j}\) among the contigs in \(\Contigs(j)\) participating in the solution:
        \begin{equation}
          \plmbonus{j} \leq \sum_{\substack{ d \in \Contigs(j) \\ \dplm{d}{j} \geq \dplm{c}{j} }} \dplm{d}{j} x_d \quad \forall (j, c) \in \Fragments{}\times\Contigs{}(j), j \text{ is a share}
        \end{equation}
    
        \begin{notebox}
          When \(c\) is the contig with the best correction among the contigs that participate in the solution, the constraint for \(c\) becomes \(\plmbonus{j} \leq \dplm{c}{j}\). As adding this correction to the objective function raises the objective value, \(\plmbonus{j} = \dplm{c}{j}\).
        \end{notebox}
    \end{refactorbox}
\end{definition}


\begin{definition}{\PPGF{} objective function}{milp:ppgf_objective}
    The objective function aims to maximize the plasmidness:
        \begin{equation}
        \begin{split}
            TotalPlasmidness & = \sum_{i \in \Fragments{}} (\plm{i} - 0.5) x_i \\
            & + \sum_{\substack{ j \in \Fragments{} \\ j \text{ is a share} }} \plmbonus{j}
        \end{split}
        \end{equation}
    It results in the objective function:
    \begin{equation}
        \max ~ TotalPlasmidness
    \end{equation}
\end{definition}

