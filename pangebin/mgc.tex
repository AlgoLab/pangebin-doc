\subsection{The Maximum GC Probability Score problem \MGC{}}\label{sec:method:mgc}

The second binning stage consists in finding a flow maximizing the GC probability scores, by fixing the coverage score flow.
We refer to this subproblem as the \enquote{Maximum GC Probability Score} problem (\MGC{}).
In the following we describe the MILP model.

\begin{definition}{\MGC{} MILP variables}{milp:mgc:variables}
  We complete \Cref{definition:milp:mcf:variables} with the following variables:
  \begin{itemize}
    \item \(GC_b \in \Set{0, 1} \, \forall b \in K\) denoting whether the plasmid GC content is in the GC content interval \(b\) or not.
    \item \(\fraggc{i}{b} \in \Reals_{>=0} \, \forall (i, b) \in \Fragments{} \times K\) denoting whether the fragment \(i\) is active and the plasmid GC content is in the interval \(b\) or not. With the constraints it acts as a binary.
  \end{itemize}
\end{definition}

\begin{definition}{\MGC{} MILP constraints}{milp:mgc:constraints}
  To the constraints in \Cref{definition:milp:mcf:constraints} we add the following:

  \phantom{text}
  The coverage score must be near to the optimal coverage score \(\Phi{}\) found in \MCF{}:
  \begin{equation}
    \gamma_1 \Phi \leq CoverageScore \quad \gamma_1 \in (0.5, 1] % chktex 9
    \cstlabel{mgc:cst:coverage_score_near_optimal} % chktex 25
  \end{equation}

  The plasmid GC content is in exactly one GC content interval \(b \in K\):
  \begin{equation}
    \sum_{b \in K} GC_b = 1
    \cstlabel{mgc:cst:exactly_one_gc_content_interval} % chktex 25
  \end{equation}

  For each fragment \(i \in \Fragments{}\), for each GC content interval \(b \in K\), \(\fraggc{i}{b} = 1\) if and only if fragment \(i\) is active and the \(b\) is the solution GC content interval:
  \begin{align}
    \fraggc{i}{b} & \leq x_{i_t} + x_{i_h} & \forall (i, b) \in \Fragments{} \times K \cstlabel{mgc:cst:active_fragment_gc_1} \\ % chktex 25
    \fraggc{i}{b} & \leq GC_b & \forall (i, b) \in \Fragments{} \times K \cstlabel{mgc:cst:active_fragment_gc_2} \\ % chktex 25
    \fraggc{i}{b} & \geq x_{i_t} + GC_b - 1 & \forall (i, b) \in \Fragments{} \times K \cstlabel{mgc:cst:active_fragment_gc_3} \\ % chktex 25
    \fraggc{i}{b} & \geq x_{i_h} + GC_b - 1 & \forall (i, b) \in \Fragments{} \times K \cstlabel{mgc:cst:active_fragment_gc_4} % chktex 25
  \end{align}
\end{definition}

\begin{definition}{\MGC{} MILP objective function}{milp:mgc:objective}
  The objective function aims to maximize the GC probability score:
  \begin{equation}
    \max ~ GCProbabilityScore
    \objlabel{mgc:obj:max_gc_probability_score} % chktex 25 % chktex 35
  \end{equation}
  where \(
    \displaystyle GCProbabilityScore = \sum_{\substack{
        i \in \Fragments{} \\
        b \in K
    }} \gcscore{i}{b} \fraggc{i}{b}%
  \).
\end{definition}