\subsection{The Maximum Plasmidness Score problem \MPS{}}\label{sec:method:mps}

The third binning stage consists in finding a flow maximizing the plasmidness scores, by fixing both the coverage flow and the GC probability score.
We refer to this subproblem as the \enquote{Maximum Plasmidness Score} problem (\MPS{}).
In the following we describe the MILP model.

\begin{definition}{\MPS{} MILP variables}{milp:mps_variables}
  We complete \Cref{definition:milp:mgc_variables} with the following variables:
  \(\frag{i} \in [0, 1] \, \forall i \in \Fragments{}\) denoting whether fragment \(i\) is active or not. With the constraint it acts as a binary.
\end{definition}

\begin{definition}{\MGC{} MILP constraints}{milp:mps_constraints}
  To the constraints in \Cref{definition:milp:mgc_constraints} we add the following:

  The total GC probability score value must be near to the total GC probability score value \(\Psi{}\) found in \MGC{}:
  \begin{equation}
    \gamma_2 \Psi \leq GCProbabilityScore \quad \gamma_2 \in (0.5, 1]
  \end{equation}

  The fragment is active if at least one of its extremities is active:
  \begin{align}
    \frag{i} \leq x_{i_t} + x_{i_h} \quad \forall i \in \Fragments{} \\
    x_{i_t} \leq \frag{i} \quad \forall i \in \Fragments{} \\
    x_{i_h} \leq \frag{i} \quad \forall i \in \Fragments{}
  \end{align}
\end{definition}

\begin{definition}{\MGC{} MILP objective function}{milp:mps_objective}
  \begin{newfeatbox}
    Here the plasmidness either act as a bonus if they are more than 0.5, or a penalty if they are less than 0.5.
  \end{newfeatbox}
  The objective function aims to maximize the plasmidness scores:
  \begin{equation}
    \max ~ PlasmidnessScore
  \end{equation}
  where \(\displaystyle PlasmidnessScore = \sum_{i \in \Fragments{}} 2 (\plm{i} - 0.5)\frag{i}\).
\end{definition}