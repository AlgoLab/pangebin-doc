\subsection{The Maximum Gene Density problem \MGD{}}\label{meth:max_gene_density}

The third binning stage consists in finding a flow maximizing the gene densities, by fixing both the coverage flow and the GC probability score.
We refer to this subproblem as the \enquote{Maximum Gene Density} problem (\MGD{}).
In the following we describe the MILP model.

\begin{definition}{\MGD{} MILP variables}{milp:mgd_variables}
    The variables are the same as in \Cref{definition:milp:mgc_variables} (itself a superset of variables in \Cref{definition:milp:mcf_variables}.
\end{definition}


\begin{definition}{\MGC{} MILP constraints}{milp:mgd_constraints}
    To the constraints in \Cref{definition:milp:mgc_constraints} we add the following:

    The total GC probability score value must be near to the total GC probability score value \(\Psi\) found in \MGC{}:
    \begin{equation}
        \gamma_2 \Psi \leq GCProbabilityScore \quad \gamma_2 \in (0.5, 1]
    \end{equation}
\end{definition}

\begin{definition}{\MGC{} MILP objective function}{milp:mgd_objective}
    The objective function aims to maximize the gene densities:
    \begin{equation}
        \max ~ GeneDensity
    \end{equation}
    where \(\displaystyle GeneDensity = \sum_{i \in \Fragments{}} \gd{i} x_i\).
\end{definition}