\subsection{The Maximum Gene Density problem \MGD{}}\label{meth:max_gene_density}

The third binning stage consists in finding a flow maximizing the gene densities, by fixing both the coverage flow and the GC probability score.
We refer to this subproblem as the \enquote{Maximum Gene Density} problem (\MGD{}).
In the following we describe the MILP model.

\begin{definition}{\MGD{} MILP variables}{milp:mgd_variables}
  We complete \Cref{definition:milp:mgc_variables} with the following variables:
  \(\frag{i} \in [0, 1] \, \forall i \in \Fragments{}\) denoting whether fragment \(i\) is active or not. With the constraint it acts as a binary.
\end{definition}

\begin{definition}{\MGC{} MILP constraints}{milp:mgd_constraints}
  To the constraints in \Cref{definition:milp:mgc_constraints} we add the following:

  The total GC probability score value must be near to the total GC probability score value \(\Psi{}\) found in \MGC{}:
  \begin{equation}
    \gamma_2 \Psi \leq GCProbabilityScore \quad \gamma_2 \in (0.5, 1]
  \end{equation}

  The fragment is active if at least one of its extremities is active:
  \begin{align}
    \frag{i} \leq x_{i_t} + x_{i_h} \quad \forall i \in \Fragments{} \\
    x_{i_t} \leq \frag{i} \quad \forall i \in \Fragments{} \\
    x_{i_h} \leq \frag{i} \quad \forall i \in \Fragments{}
  \end{align}
\end{definition}

\begin{definition}{\MGC{} MILP objective function}{milp:mgd_objective}
  The objective function aims to maximize the gene densities:
  \begin{equation}
    \max ~ GeneDensity
  \end{equation}
  where \(\displaystyle GeneDensity = \sum_{i \in \Fragments{}} \gd{i} \frag{i}\).
\end{definition}