\subsubsection{The Maximum Binning Score problem \MBS{}}\label{sec:pbf_iterbin:binlab:mbs}

The first binning stage consists in finding a flow maximizing the use of fragment coverage and the plasmidness.
We refer to this subproblem as the \enquote{Maximum Binning Score} problem (\MBS{}).
As a coarse-grain strategy, it consists in finding a connected component explaining both the coverages and the plasmidness.
Below we describe the MILP model.

\begin{table}[h!]
  \centering
  \tablecaption{\MBS{} MILP constraints}{}\label{tab:binlab:mbs:cst}
  \begin{tabular}{@{}ll@{}}
    \toprule
    \tabhtxt{Category} & \tabhtxt{Constraints} \\
    \midrule
    Decision & \Cref{%
      pbf_iterbin:milp:cst:dvar:active_extremity_implies_active_fragment,%
      pbf_iterbin:milp:cst:dvar:active_fragment_implies_active_extremity,%
      pbf_iterbin:milp:cst:dvar:active_link_arc_active_fragments,%
      pbf_iterbin:milp:cst:dvar:active_vertex_active_incoming_arcs%
    } \\
    Flow & \Cref{%
      pbf_iterbin:milp:cst:flow:arc_flow_at_most_coverage,%
      pbf_iterbin:milp:cst:flow:inflow_at_most_coverage,%
      pbf_iterbin:milp:cst:flow:inflow_equals_outflow_forward,%
      pbf_iterbin:milp:cst:flow:inflow_equals_outflow_reverse,%
      pbf_iterbin:milp:cst:flow:total_flow_source,%
      pbf_iterbin:milp:cst:flow:total_flow_sink,%
      pbf_iterbin:milp:cst:flow:arc_flow_at_least_total_flow_1,%
      pbf_iterbin:milp:cst:flow:arc_flow_at_least_total_flow_2,%
      pbf_iterbin:milp:cst:flow:arc_flow_at_least_total_flow_3%
    } \\
    Connected component & \Cref{%
      pbf_iterbin:milp:cst:ccomp:one_outgoing_arc_from_source,%
      pbf_iterbin:milp:cst:ccomp:positive_flow_incident_vertices_in_component,%
      pbf_iterbin:milp:cst:ccomp:depth_of_tree_source,%
      pbf_iterbin:milp:cst:ccomp:depth_of_tree_incident_arcs,%
      pbf_iterbin:milp:cst:ccomp:tree_arc_active,%
      pbf_iterbin:milp:cst:ccomp:number_of_active_vertices%
    } \\
    GC & --- \\
    Miscellaneous & \Cref{%
      pbf_iterbin:milp:cst:misc:min_cumulative_fragment_length%
    } \\
    \bottomrule
  \end{tabular}
\end{table}

\begin{definition}{\MBS{} MILP objective function}{pbf_iterbin:binlab:mbs:obj}
  \begin{newfeatbox}
    Here we maximize the coverage scores, not only the coverage flow.
  \end{newfeatbox}
  The objective function aims to maximize the total coverage scores:
  \begin{equation}
    \max ~ BinningScore
    \objlabel{pbf_iterbin:binlab:mbs:obj:max_coverage_score} % chktex 25 % chktex 35
  \end{equation}
  where \(
    \displaystyle BinningScore = %
    \sum_{i \in \Fragments{}} \fzeta{i} \parenth[\big]{ inflow(i) - (\cov{i}\frag{i} - inflow(i)) + \plm{i} inflow(i)}
  \) and \( \displaystyle \fzeta{i} = \frac{ |i| }{ \max_{j \in \Fragments{}} |j| } \).

  \begin{ideabox}
    Perhaps we should use the coverage as a penalty and only on positive plasmidness, and replace \fzeta{} by the fragment length, i.e.:
    \[
      BinningScore =  \sum_{
        \substack{
          i \in \Fragments{} \\
          \plm{i} > 0
        }
      } |i| \parenth*{ inflow(i) - \cov{i} }
      + \sum_{i \in \Fragments{}} |i| \plm{i} inflow(i)
    \]
  \end{ideabox}
\end{definition}