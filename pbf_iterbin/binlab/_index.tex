\subsection{Approach \texttt{binlab}}\label{sec:pbf_iterbin:binlab}

The \texttt{binlab} approach tackles the binning of the fragments with a coarse-to-fine-grain approach in three stages (each one refering to an optimization problem):
The ideas are quite similar to those in the \texttt{decomp} approach (\Cref{sec:pbf_iterbin:decomp}), except that here we combine the use of the coverage and the plasmidness in one objective function.

\begin{enumerate}[label=\roman*.]
  \item Find the maximum binning score (\MBS{} problem, \Cref{sec:pbf_iterbin:binlab:mbs})
  \item Find the maximum labelling score (\MLS{} problem, \Cref{sec:pbf_iterbin:binlab:mls})
  \item Refine the solution (\MRBS{} problem, \Cref{sec:pbf_iterbin:binlab:mrbs})
\end{enumerate}

\Cref{algo:pbf_iterbin:binlab} summarizes the \texttt{binlab} approach.

\begin{tcbalgo}{\texttt{binlab} approach}{pbf_iterbin:binlab}
  \begin{algorithmic}[1]
    \Require{%
      A network graph \(N\) as defined in \Cref{sec:pbf_iterbin:network}.
    }
    \Ensure{%
      Extract the best bin from the network graph.
    }
    \Function{find\_best\_bin}{\( N \)}
    \State{} Init \MBS{} and solve it
    \If{\MBS{} is unfeasible}
    \State{} \Return{}unfeasible message
    \EndIf{}
    \State{} Init \MLS{} from \MBS{} result and solve it
    \If{\MLS{} is unfeasible}
    \State{} \Return{}unfeasible message
    \EndIf{}
    \State{} Init \MRBS{} from \MLS{} result and solve it
    \If{\MRBS{} is unfeasible}
    \State{} \Return{}unfeasible message
    \EndIf{}
    \State{} \Return{}\MRBS{} bin result (defined by the MILP solution variables induced subgraph)
    \EndFunction{}
  \end{algorithmic}
\end{tcbalgo}

Between each step, we fix the objective value of the previous step.
A final step refines the solution found in the \MLS{} step (\Cref{sec:pbf_iterbin:binlab:mrbs}).

\subsubsection{The Maximum Binning Score problem \MBS{}}\label{sec:pbf_iterbin:binlab:mbs}

The first binning stage consists in finding a flow maximizing the use of fragment coverage and the plasmidness.
We refer to this subproblem as the \enquote{Maximum Binning Score} problem (\MBS{}).
As a coarse-grain strategy, it consists in finding a connected component explaining both the coverages and the plasmidness.
Below we describe the MILP model.

\begin{table}[h!]
  \centering
  \tablecaption{\MBS{} MILP constraints}{}\label{tab:binlab:mbs:cst}
  \begin{tabular}{@{}ll@{}}
    \toprule
    \tabhtxt{Category} & \tabhtxt{Constraints} \\
    \midrule
    Decision & \zcref[S]{%
      pbf_iterbin:milp:cst:dvar:active_vertex_implies_active_contig,%
      pbf_iterbin:milp:cst:dvar:active_contig_implies_active_vertex,%
      pbf_iterbin:milp:cst:dvar:active_arc_active_vertices,%
      pbf_iterbin:milp:cst:dvar:active_vertex_active_incoming_arcs%
    } \\
    Flow & \zcref[S]{%
      pbf_iterbin:milp:cst:flow:arc_flow_at_most_coverage,%
      pbf_iterbin:milp:cst:flow:inflow_at_most_coverage,%
      pbf_iterbin:milp:cst:flow:flow_conservation,%
      pbf_iterbin:milp:cst:flow:total_flow_source,%
      pbf_iterbin:milp:cst:flow:total_flow_sink,%
      pbf_iterbin:milp:cst:flow:total_flow_strictly_positive,%
      pbf_iterbin:milp:cst:flow:arc_flow_at_least_total_flow_1,%
      pbf_iterbin:milp:cst:flow:arc_flow_at_least_total_flow_2,%
      pbf_iterbin:milp:cst:flow:arc_flow_at_least_total_flow_3%
    } \\
    Connected component & \zcref[S]{%
      pbf_iterbin:milp:cst:ccomp:one_outgoing_arc_from_source,%
      pbf_iterbin:milp:cst:ccomp:positive_flow_incident_vertices_in_component,%
      pbf_iterbin:milp:cst:ccomp:depth_of_tree_source,%
      pbf_iterbin:milp:cst:ccomp:depth_of_tree_incident_arcs,%
      pbf_iterbin:milp:cst:ccomp:tree_arc_active,%
      pbf_iterbin:milp:cst:ccomp:number_of_active_vertices%
    } \\
    GC & --- \\
    Miscellaneous & \zcref[S]{%
      pbf_iterbin:milp:cst:misc:min_cumulative_contig_length%
    } \\
    \bottomrule
  \end{tabular}
\end{table}

\begin{definition}{\MBS{} MILP objective function}{pbf_iterbin:binlab:mbs:obj}
  The objective function aims to maximize the total coverage scores:
  \begin{Objective}
    \begin{equation}
      \max ~ BinningScore
      \label{pbf_iterbin:binlab:mbs:obj:max_coverage_score} % chktex 25 % chktex 35
    \end{equation}
  \end{Objective}
  where \(
    \displaystyle BinningScore = %
    \sum_{i \in \Fragments{}} \fzeta{i} \parenth[\big]{ inflow(i) - (\cov{i}\frag{i} - inflow(i)) + \plm{i} inflow(i)}
  \) and \( \displaystyle \fzeta{i} = \frac{ |i| }{ \max_{j \in \Fragments{}} |j| } \).

  \begin{ideabox}
    Perhaps we should use the coverage as a penalty and only on positive plasmidness, and replace \fzeta{} by the fragment length, i.e.:
    \[
      BinningScore =  \sum_{
        \substack{
          i \in \Fragments{} \\
          \plm{i} > 0
        }
      } |i| \parenth*{ inflow(i) - \cov{i} }
      + \sum_{i \in \Fragments{}} |i| \plm{i} inflow(i)
    \]
  \end{ideabox}
\end{definition}
\subsubsection{The Maximum Labelling Score problem \MLS{}}\label{sec:pbf_iterbin:binlab:mls}

The second binning stage consists in finding the best GC bin label, for a bin with a binning score lower bound.
We refer to this subproblem as the \enquote{Maximum Labelling Score} problem (\MLS{}).
In the following we describe the MILP model.

\begin{table}[!htbp]
  \centering
  \tablecaption{\MLS{} MILP constraints}{}\label{tab:binlab:mls:cst}
  \begin{tabular}{@{}ll@{}}
    \toprule
    \tabhtxt{Category} & \tabhtxt{Constraints} \\
    \midrule
    \multicolumn{2}{@{}l@{}}{Constraints in \zcref[S]{tab:binlab:mbs:cst}} \\
    \addlinespace
    GC & \zcref[S]{%
      pbf_iterbin:milp:cst:gc:exactly_one_gc_content_interval,%
      pbf_iterbin:milp:cst:gc:active_fragment_gc_1,%
      pbf_iterbin:milp:cst:gc:active_fragment_gc_2,%
      pbf_iterbin:milp:cst:gc:active_fragment_gc_3,%
      pbf_iterbin:milp:cst:gc:active_fragment_gc_4%
    } \\
    \MLS{} dedicated & \zcref[S]{pbf_iterbin:binlab:mls:cst:fix_mbs_obj} \\
    \bottomrule
  \end{tabular}
\end{table}

In addition to the constraints in \zcref[S]{tab:binlab:mls:cst}, we constraint the coverage score to be near to the optimal coverage score \(\Phi{}\) previously found in \MBS{}:
%
\begin{Constraint}
  \begin{equation}
    \Phi - (1 - \gamma_1) | \Phi | \leq BinningScore \quad \gamma_1 \in (0.5, 1] % chktex 9
    \cstlabel{pbf_iterbin:binlab:mls:cst:fix_mbs_obj} % chktex 25
  \end{equation}
\end{Constraint}

\begin{definition}{\MLS{} MILP objective function}{pbf_iterbin:binlab:mls:obj}

  The objective function aims to maximize the GC score:
  %
  \begin{Objective}
    \begin{equation}
      \max ~ LabellingScore
      \objlabel{pbf_iterbin:binlab:mls:obj:max_labelling_score} % chktex 25 % chktex 35
    \end{equation}
  \end{Objective}
  %
  where \(
    \displaystyle LabellingScore = \sum_{\substack{
        i \in \Fragments{} \\
        b \in K
    }} \gcscore{i}{b} \fraggc{i}{b}%
  \).

\end{definition}
\subsubsection{The Maximum Refining the \MLS{} solution}\label{sec:pbf_iterbin:binlab:mrbs}

The solution set \(S_\MLS{}\) for \MLS{} contains solutions with the same value of \(LabellingScore\) and different values for \(BinningScore\) because of \zcref[S]{pbf_iterbin:binlab:mls:cst:fix_mbs_obj}.
The final refinement stage consists in fixing the value of \(LabellingScore\) and finding the subset of \(S_\MLS{}\) maximizing \(BinningScore\).
It results in a last MILP formulation \(\MRBS{}\):

\begin{table}[h!]
  \centering
  \tablecaption{\MRBS{} MILP constraints}{}\label{tab:binlab:mrbs:cst}
  \begin{tabular}{@{}ll@{}}
    \toprule
    \tabhtxt{Category} & \tabhtxt{Constraints} \\
    \midrule
    \multicolumn{2}{@{}l@{}}{Constraints in \zcref[S]{tab:binlab:mls:cst}} \\
    \addlinespace
    \MRBS{} dedicated & \zcref[S]{pbf_iterbin:binlab:mrbs:cst:fix_mls_obj} \\
    \bottomrule
  \end{tabular}
\end{table}

The labelling score must be equal to the labelling score \(\Lambda{}\) found in \MLS{}:
\begin{Constraint}
  \begin{equation}
    LabellingScore = \Lambda
    \label{pbf_iterbin:binlab:mrbs:cst:fix_mls_obj} % chktex 25
  \end{equation}
\end{Constraint}

\begin{definition}{\MRBS{} MILP objective function}{pbf_iterbin:binlab:mrbs:obj}
  The objective function aims to maximize the binning score:
  \begin{Objective}
    \begin{equation}
      \max ~ BinningScore
      \label{pbf_iterbin:binlab:mrbs:obj:max_binning_score} % chktex 25 % chktex 35
    \end{equation}
  \end{Objective}
\end{definition}
