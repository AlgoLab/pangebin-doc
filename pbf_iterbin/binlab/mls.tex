\subsubsection{The Maximum Labelling Score problem \MLS{}}\label{sec:pbf_iterbin:binlab:mls}

The second binning stage consists in finding the best GC bin label, for a bin with a binning score lower bound.
We refer to this subproblem as the \enquote{Maximum Labelling Score} problem (\MLS{}).
In the following we describe the MILP model.

\begin{table}[!htbp]
  \centering
  \tablecaption{\MLS{} MILP constraints}{}\label{tab:binlab:mls:cst}
  \begin{tabular}{@{}ll@{}}
    \toprule
    \tabhtxt{Category} & \tabhtxt{Constraints} \\
    \midrule
    \multicolumn{2}{@{}l@{}}{Constraints in \Cref{tab:binlab:mbs:cst}} \\
    \addlinespace
    GC & \Cref{%
      pbf_iterbin:milp:cst:gc:exactly_one_gc_content_interval,%
      pbf_iterbin:milp:cst:gc:active_fragment_gc_1,%
      pbf_iterbin:milp:cst:gc:active_fragment_gc_2,%
      pbf_iterbin:milp:cst:gc:active_fragment_gc_3,%
      pbf_iterbin:milp:cst:gc:active_fragment_gc_4%
    } \\
    \MLS{} dedicated & \Cref{pbf_iterbin:binlab:mls:cst:fix_mbs_obj} \\
    \bottomrule
  \end{tabular}
\end{table}

In addition to the constraints in \Cref{tab:binlab:mls:cst}, we constraint the coverage score to be near to the optimal coverage score \(\Phi{}\) previously found in \MBS{}:
%
\begin{Constraint}
  \begin{equation}
    \Phi - (1 - \gamma_1) | \Phi | \leq BinningScore \quad \gamma_1 \in (0.5, 1] % chktex 9
    \cstlabel{pbf_iterbin:binlab:mls:cst:fix_mbs_obj} % chktex 25
  \end{equation}
\end{Constraint}

\begin{definition}{\MLS{} MILP objective function}{pbf_iterbin:binlab:mls:obj}

  The objective function aims to maximize the GC score:
  %
  \begin{Objective}
    \begin{equation}
      \max ~ LabellingScore
      \objlabel{pbf_iterbin:binlab:mls:obj:max_labelling_score} % chktex 25 % chktex 35
    \end{equation}
  \end{Objective}
  %
  where \(
    \displaystyle LabellingScore = \sum_{\substack{
        i \in \Fragments{} \\
        b \in K
    }} \gcscore{i}{b} \fraggc{i}{b}%
  \).

\end{definition}