\subsection{The network graph}\label{sec:pbf_iterbin:network}

All the MILP flow-based approaches are built on the top of a flow network:

\begin{itemize}
  \item We add two vertices, \source{} and \sink{}, that are respectively the source and the sink vertices.
  \item The set \(\VSeeds{}\) contains the vertices associated with the seed contigs in \ContigSeeds{}.
    The source goes into each vertex in \(\VSeeds{}\), and each vertex in \(V_{\Contigs{}}\) goes into the sink.
\end{itemize}

\begin{notebox}
  We discuss in \zcref[S]{sec:pbf_iterbin:conclusions} the connection of the source to only the seeds.
\end{notebox}

\begin{definition}{Network graph}{pbf_iterbin:network_graph}
  Let \(N = (V, A, \source{}, \sink{}, \cov{\Contigs{}}, \gcscore{\Contigs{}}{K}, \plm{\Contigs{}})\) be the network graph, where:

  \begin{itemize}
    \item \( V = V_\Contigs{} \cup \Set*{\source{}, \sink{}} \) is the set of vertices with the source \(\source{}\) and the sink \(\sink{}\);
    \item \( A = A_\Links{} \cup \Set{(\source, v) \given v \in \VSeeds{}} \cup \Set{(v, \sink) \given v \in V} \) is the set of link-arcs augmented with the source-arcs and the sink-arcs;
    \item \( \cov{\Contigs{}} \colon \Contigs{} \to \Reals{}_{> 0} \) is the coverage function;
    \item \( \gcscore{\Contigs{}}{K} \colon \Contigs{} \times K \to [-1, 1] \) is the GC score function;
    \item \( \plm{\Contigs{}} \colon \Contigs{} \to [-1, 1] \) is the plasmidness function.
  \end{itemize}
\end{definition}
