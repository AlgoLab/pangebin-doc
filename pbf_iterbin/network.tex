\subsection{The network graph}\label{sec:pbf_iterbin:network}

All the MILP flow-based approaches are built on the top of a flow network:

\begin{itemize}
  \item To keep the direction of the flow along the link-edges in \(E_\Links{}\), we transform each link-edge to two link-arcs with the function \(etoa \colon \Set*{u, v} \mapsto \Set*{(u, v), (v, u)}\).
  \item We add two vertices, \(s\) and \(t\) that are respectively the source and the sink vertices.
  \item The set \(\VSeed{}\) contains the vertices associated with the seed fragments in \SeedFrags{}.
    The source goes into each vertex in \(\VSeed{}\), and each vertex in \(V_t \cup V_h\) goes into the sink.
\end{itemize}

\begin{notebox}
  % #TODO complete this note
  We discuss in section??? the connection of the source to only the seeds.
\end{notebox}

\begin{definition}{Network graph}{pbf_iterbin:network_graph}
  Let \(N = (V, A_\Fragments{} \cup A_\Links{}, s, t, \cov{\Fragments{}}, \gcscore{\Fragments{}}{K}, \plm{\Fragments{}})\) be the network graph, where:

  \begin{itemize}
    \item \( V = V_t \cup V_h \cup \Set*{s, t} \) is the set of vertices with the source \(s\) and the sink \(t\);
    \item \( A_\Fragments{} = \bigcup_{e \in E_\Fragments{}} etoa(e) \) is the set of fragment-arcs;
    \item \( A_\Links{} = \bigcup_{e \in E_\Links{}} etoa(e) \cup \Set{(s, v) \given v \in \VSeed{}} \cup \Set{(v, t) \given v \in V} \) is the set of link-arcs augmented with the seed-arcs and the sink-arcs;
    \item \( \cov{\Fragments{}} \colon \Fragments{} \to \Reals{}_{> 0} \) is the coverage function;
    \item \( \gcscore{\Fragments{}}{K} \colon \Fragments{} \times K \to [-1, 1] \) is the GC score function;
    \item \( \plm{\Fragments{}} \colon \Fragments{} \to [-1, 1] \) is the plasmidness function.
  \end{itemize}
\end{definition}
