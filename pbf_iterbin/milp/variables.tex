\zcref[S]{tab:pbf_iterbin:milp:variables} lists all the variables used in the MILP models.

\begin{center}
  \begin{xltabular}{\linewidth}{%
      @{}%
      l% Name of the variable
      l% Codomain of the variable
      l% Relaxed codomain of the variable
      l% Application set of the variable
      X% Meaning
      @{}%
    }
    \longtablecaption{MILP variables}{%
      Each of the following variables participates in at least one of the MILP model.
      Some of them participate in all the model, others are necessary for only one model.
      For the ease of read, we categorize the variables in three categories.
      Each section describing a model precise which variables are participating for each category.
    }\label{tab:pbf_iterbin:milp:variables}\\

    \toprule
    {\tabhtxt{Variable}} & {\tabhtxt{Domain}} & \multicolumn{2}{c}{\tabhtxt{Codomains}} & {\tabhtxt{Meaning}} \\
    \cmidrule(lr){3-4}
    & & {\tabhtxt{Practice}} & {\tabhtxt{Relax}} & \\
    \midrule

    \endfirsthead%

    \multicolumn{5}{@{}l}{\sffamily\footnotesize{\zcref[S]{tab:pbf_iterbin:milp:variables}, continued}} \\
    \addlinespace
    \toprule
    {\tabhtxt{Variable}} & {\tabhtxt{Domain}} & \multicolumn{2}{c}{\tabhtxt{Codomains}} & {\tabhtxt{Meaning}} \\
    \cmidrule(lr){3-4}
    & & {\tabhtxt{Practice}} & {\tabhtxt{Relax}} & \\
    \midrule

    \endhead%

    \midrule

    \multicolumn{5}{r@{}}{\sffamily\footnotesize{(the table continues on the next page)}}
    \endfoot%

    \bottomrule
    \endlastfoot%

    \multicolumn{5}{@{}l@{}}{\tabhtxt{Decision variables}} \\
    \addlinespace
    \(x_v\) & \(V \setminus \Set*{\source{}, \sink{}}\) & \(\Set{0, 1}\) & \(\Reals_{\geq 0}\) & Denoting whether the vertex \(v\) is active or not. \\
    \addlinespace
    \(y_a\) & \(A\) & \(\Set{0, 1}\) & & Denoting whether the arc \(a\) is active or not. \\
    \addlinespace
    \(\contig{i}\) & \(\Contigs{}\) & \(\Set{0, 1}\) & \([0, 1]\) & Denoting whether contig \(i\) is active or not. \\
    \addlinespace
    \(GC_k\) & \(K\) & \(\Set{0, 1}\) & & Denoting whether the plasmid GC content is in the GC content interval \(k\) or not. \\
    \addlinespace
    \(\contiggc{i}{k}\) & \(\Contigs{} \times K\) & \(\Set{0, 1}\) & \(\Reals_{\geq 0}\) & Denoting whether the contig \(i\) is active and the plasmid GC content is in the interval \(k\) or not. \\
    %
    \addlinespace
    \multicolumn{5}{@{}l@{}}{\tabhtxt{Flow variables}} \\
    \addlinespace
    \(f_a\) & \(A\) & \(\Reals_{\geq 0}\) & & The flow amount passing through the arc \(a\). \\
    \addlinespace
    \(F\) & & \(\Reals_{\geq 0}\) & & The total flow. \\
    \addlinespace
    \(F_a\) & \(A\) & \(\Reals_{\geq 0}\) & & Intermediary variable to force the positive flows to be lower bound by the total flow. \\
    \addlinespace
    \(\inflowgc{i}{k}\) & \(\Contigs{} \times K\) & \(\Reals_{\geq 0}\) & & Flow amount passing through the contig \(i\) when the plasmid GC content is in the interval \(k\). \\
    %
    \addlinespace
    \multicolumn{5}{@{}l@{}}{\tabhtxt{Connected component variables}} \\
    \addlinespace
    \(\alpha{}\) & & \(\Set{ 0 \ldots |V| }\) & \(\Reals{}\) & Is the number of \emph{oriented} fragments in the solution connected graph, plus the source and the sink. \\
    \addlinespace
    \(\beta_a\) & \(A\) & \(\Set{- |V| \ldots 0}\) & \(\Reals_{\leq 0}\) & Is strictly negative if the link-arc \(a\) participates in one of the solution subgraph's exploration tree. The absolute value corresponds to the depth of the subtree with root the successor. \\
  \end{xltabular}
\end{center}
