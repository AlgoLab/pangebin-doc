Here we describe and categorize the set of constraints composing the different MILP models.
In what follow, we define either a fragment, a contig or an arc to be \enquote{active} if it participates in the solution, i.e.\ the flow passes through it.

\paragraph{Decision variables relationships}

A fragment is active if, and only if at least one of its extremity vertices is active:
\begin{align}
  x_v & \leq \frag{i} & \forall v \in V \setminus \Set*{s, t}, i = vfrag(v) \cstlabel{pbf_iterbin:milp:cst:dvar:active_extremity_implies_active_fragment} \\ % chktex 25
  \frag{i} & \leq x_u + x_v & \forall i \in \Fragments{}, \{u, v\} \in \EFrags{} \cstlabel{pbf_iterbin:milp:cst:dvar:active_fragment_implies_active_extremity} % chktex 25
\end{align}

The fragments involved in an active link-arc must also be active:
\begin{equation}
  y_{uv} \leq
  \begin{cases}
    x_v & \text{if \(u = s\)} \\
    x_u & \text{if \(v = t\)} \\
    \min\Set{x_u, x_v} & \text{otherwise}
  \end{cases} \quad \forall (u, v) \in A_\Links{} %
  \cstlabel{pbf_iterbin:milp:cst:dvar:active_link_arc_active_fragments} % chktex 25
\end{equation}

An active vertex implies at least one active link-arc incoming to it:
\begin{questionbox}
  Are these constraints necessary?
\end{questionbox}
\begin{equation}
  x_v \leq \sum_{(u, v) \in A_\Links{}} y_{u v} \quad \forall v \in V \setminus \Set{s, t} \cstlabel{pbf_iterbin:milp:cst:dvar:active_vertex_active_incoming_arcs} % chktex 25
\end{equation}

\paragraph{Flow constraints}

The flow through a link-arc \(a \in A_\Links{}\) is non-zero if \(a\) is active and cannot exceed its capacity:
\begin{align}
  f_{uv} \leq
  \begin{cases}
    \cov{j} y_{uv} & \text{if \(u = s\)} \\
    \cov{i} y_{uv} & \text{if \(v = t\)} \\
    \min\Set*{\cov{i}, \cov{j}} y_{uv} & \text{otherwise}
  \end{cases} &
  \begin{split}
    \forall (u, v) \in A_\Links{} \\
    i = vfrag(u) \\
    j = vfrag(v)
  \end{split} \cstlabel{pbf_iterbin:milp:cst:flow:arc_flow_at_most_coverage} % chktex 25
\end{align}

The cumulative flow through a fragment \(i \in \Fragments{}\) cannot exceed its read coverage:
\begin{equation}
  inflow(i) \leq \cov{i} \quad \forall i \in \Fragments{} \cstlabel{pbf_iterbin:milp:cst:flow:inflow_at_most_coverage} % chktex 25
\end{equation}
Where \(\displaystyle inflow(i) = \sum_{(u, i_t) \in A_\Links{}} f_{ui_t} + \sum_{(u, i_h) \in A_\Links{}} f_{ui_h}\) where \(i_t\) and \(i_h\) respectively stand for the tail and the head vertices of fragment \(i\).

The cumulative flow into a fragment \(i\) should be equal to the cumulative flow out of it. Same for the reverse \(i^-\):
\begin{align}
  \sum_{(u, i_t) \in A_\Links{}} f_{u i_t} & = \sum_{(i_h, w) \in A_\Links{}} f_{i_h w} & \forall i \in \Fragments{} \cstlabel{pbf_iterbin:milp:cst:flow:inflow_equals_outflow_forward} \\ % chktex 25
  \sum_{(u, i_h) \in A_\Links{}} f_{u i_h} & = \sum_{(i_t, w) \in A_\Links{}} f_{i_t w} & \forall i \in \Fragments{} \cstlabel{pbf_iterbin:milp:cst:flow:inflow_equals_outflow_reverse} % chktex 25
\end{align}

The total flow value \(F\) equals to the flow out of \(s\) and into \(t\):
\begin{align}
  F & = \sum_{(s, v) \in A_\Links{}} f_{sv} \cstlabel{pbf_iterbin:milp:cst:flow:total_flow_source} \\ % chktex 25
  F & = \sum_{(v, t) \in A_\Links{}} f_{vt} \cstlabel{pbf_iterbin:milp:cst:flow:total_flow_sink} % chktex 25
\end{align}

The total flow is strictly positive:
\begin{equation}
  F \geq \epsilon_F \quad \epsilon_F > 0 \cstlabel{pbf_iterbin:milp:cst:flow:total_flow_strictly_positive} % chktex 25
\end{equation}

An active link-arc has a flow at least \(F\).
\begin{align}
  F_a & \geq F - (1 - y_a) \max_{i \in \SeedFrags{}}\Set{\cov{i}} & \forall a \in A_\Links{} \cstlabel{pbf_iterbin:milp:cst:flow:arc_flow_at_least_total_flow_1} \\ % chktex 25
  F_a & \leq F & \forall a \in A_\Links{} \cstlabel{pbf_iterbin:milp:cst:flow:arc_flow_at_least_total_flow_2} \\ % chktex 25
  F_a & \leq f_a & \forall a \in A_\Links{} \cstlabel{pbf_iterbin:milp:cst:flow:arc_flow_at_least_total_flow_3} % chktex 25
\end{align}
\begin{infobox}
  These constraints do not force the arc flows to be a multiple of \(F\), see~\Cref{proposition:inflow_is_not_multiple_of_total_flow}.
\end{infobox}
\begin{missingproofbox}
  The above constraints minimize the number of out link-arcs for each fragment.
\end{missingproofbox}
\begin{questionbox}
  What is the meaning for a fragment to have a cumulative flow that is not a multiple of \(F\)?
  By keeping the flow real, can we smartly force the cumulative flow to be a multiple of \(F\)?
\end{questionbox}

\paragraph{Connected component constraints}

Exactly one link-arc outs of \(s\) is part of the solution (necessary to ensure the solution induced subgraph has only one connected component):
\begin{equation}
  \sum_{(s, v) \in A_\Links{}} y_{sv} = 1 \cstlabel{pbf_iterbin:milp:cst:ccomp:one_outgoing_arc_from_source} % chktex 25
\end{equation}

A positive flow implies the two incident vertices are in the component:
\begin{equation}
  1 - y_{uv} \geq
  \begin{cases}
    1 - x_v & \text{if \(u = s\)} \\
    x_u - 1 & \text{if \(v = t\)} \\
    x_u - x_v & \text{otherwise}
  \end{cases}
  \quad \forall (u, v) \in A_\Links{}
  \cstlabel{pbf_iterbin:milp:cst:ccomp:positive_flow_incident_vertices_in_component} % chktex 25
\end{equation}
\begin{questionbox}
  potentially redundant with \Cref{pbf_iterbin:milp:cst:dvar:active_vertex_active_incoming_arcs}
\end{questionbox}

The cumulative depth of the subtree in the exploration tree from the source equals at least the number of active vertices minus 1.
\begin{align}
  \alpha + \sum_{(s, w) \in A_\Links{}} \beta_{sw} \leq 1 \cstlabel{pbf_iterbin:milp:cst:ccomp:depth_of_tree_source} % chktex 25
\end{align}

Two incident arcs in the exploration tree are distanced by one.
\begin{align}
  \sum_{(v, w) \in A_\Links{}} \beta_{vw} - \sum_{(u, v) \in A_\Links{}} \beta_{uv} \leq 1 \quad \forall v \in V \setminus \Set*{s}
  \cstlabel{pbf_iterbin:milp:cst:ccomp:depth_of_tree_incident_arcs} % chktex 25
\end{align}

An arc participates in the exploration tree implies the arc is active.
\begin{align}
  \beta_a \geq - y_a |V| \quad \forall a \in A_\Links{}
  \cstlabel{pbf_iterbin:milp:cst:ccomp:tree_arc_active} % chktex 25
\end{align}

The number of active vertices equals \(\alpha{}\) (SAT constraint):
\begin{align}
  2 + \sum_{v \in V \setminus \Set*{s, t}} x_v = \alpha
  \cstlabel{pbf_iterbin:milp:cst:ccomp:number_of_active_vertices} % chktex 25
\end{align}

\paragraph{GC constraints}
%
The following constraints define the GC content labelling of a bin.

\phantom{text}

Only one GC content interval \(b \in K\) labels the bin:
\begin{equation}
  \sum_{b \in K} GC_b = 1
  \cstlabel{pbf_iterbin:milp:cst:gc:exactly_one_gc_content_interval} % chktex 25
\end{equation}

For each fragment \(i \in \Fragments{}\), for each GC content interval \(b \in K\), \(\fraggc{i}{b} = 1\) if and only if fragment \(i\) is active and the bin is labelled by the GC content interval \(b\):
\begin{align}
  \fraggc{i}{b} & \leq x_{i_t} + x_{i_h} & \forall (i, b) \in \Fragments{} \times K \cstlabel{pbf_iterbin:milp:cst:gc:active_fragment_gc_1} \\ % chktex 25
  \fraggc{i}{b} & \leq GC_b & \forall (i, b) \in \Fragments{} \times K \cstlabel{pbf_iterbin:milp:cst:gc:active_fragment_gc_2} \\ % chktex 25
  \fraggc{i}{b} & \geq x_{i_t} + GC_b - 1 & \forall (i, b) \in \Fragments{} \times K \cstlabel{pbf_iterbin:milp:cst:gc:active_fragment_gc_3} \\ % chktex 25
  \fraggc{i}{b} & \geq x_{i_h} + GC_b - 1 & \forall (i, b) \in \Fragments{} \times K \cstlabel{pbf_iterbin:milp:cst:gc:active_fragment_gc_4} % chktex 25
\end{align}

For each fragment \(i \in \Fragments{}\), for each GC content interval \(b \in K\), \(\inflowgc{i}{b}\) is the flow into fragment \(i\) when \(b\) is the chosen GC content interval:
\begin{align}
  \inflowgc{i}{b} & \leq \frag{i} * \cov{i} & \forall (i, b) \in \Fragments{} \times K \cstlabel{pbf_iterbin:milp:cst:gc:inflowgc_1}  \\ % chktex 25
  \inflowgc{i}{b} & \leq \frag{i} * GC_b & \forall (i, b) \in \Fragments{} \times K \cstlabel{pbf_iterbin:milp:cst:gc:inflowgc_2}  \\ % chktex 25
  \inflowgc{i}{b} & \geq inflow(i)  - (2 - \frag{i} - GC_b) \cov{i} & \forall (i, b) \in \Fragments{} \times K \cstlabel{pbf_iterbin:milp:cst:gc:inflowgc_3} % chktex 25
\end{align}

\paragraph{Plasmid constraints}

The cumulative fragment length is above a given threshold:
%
\begin{equation}
  \sum_{i \in \SeedFrags{}} |i| \geq L \quad L \geq 0 \cstlabel{pbf_iterbin:milp:cst:misc:min_cumulative_fragment_length} % chktex 25 chktex 35
\end{equation}
%
Typically, we choose \(L = 1000\).