Here we describe and categorize the set of constraints composing the different MILP models.
In what follow, we define either a contig, a vertex or an arc to be \enquote{active} if it participates in the solution, i.e.\ the flow passes through it.

\paragraph{Decision variables relationships}

A contig is active if, and only if at least one of its corresponding vertices is active:
\begin{Constraint}
  \begin{align}
    x_v & \leq \contig{i} & \forall v \in V \setminus \Set*{\source{}, \sink{}}, \quad i = vcontig(v) \label{pbf_iterbin:milp:cst:dvar:active_vertex_implies_active_contig} \\ % chktex 25
    \contig{i} & \leq x_v + x_{\rev{v}} & \forall i \in \Contigs{}, \quad (v, \rev{v}) = contigv(i) \label{pbf_iterbin:milp:cst:dvar:active_contig_implies_active_vertex} % chktex 25
  \end{align}
\end{Constraint}

The vertices involved in an active arc must also be active:
\begin{Constraint}
  \begin{equation}
    y_{uv} \leq
    \begin{cases}
      x_v & \text{if \(u = \source{}\)} \\
      x_u & \text{if \(v = \sink{}\)} \\
      \min\Set{x_u, x_v} & \text{otherwise}
    \end{cases} \quad \forall (u, v) \in A %
    \label{pbf_iterbin:milp:cst:dvar:active_arc_active_vertices} % chktex 25
  \end{equation}
\end{Constraint}

An active vertex implies at least one active arc incoming to it:
\begin{questionbox}
  Are these constraints necessary?
\end{questionbox}
\begin{Constraint}
  \begin{equation}\label{pbf_iterbin:milp:cst:dvar:active_vertex_active_incoming_arcs} % chktex 25
    x_v \leq \sum_{(u, v) \in A} y_{u v} \quad \forall v \in V \setminus \Set{\source{}, \sink{}}
  \end{equation}
\end{Constraint}

\paragraph{Flow constraints}

The flow through an arc \(a \in A\) is non-zero if \(a\) is active and cannot exceed its capacity:
\begin{Constraint}
  \begin{align}
    f_{uv} \leq
    \begin{cases}
      \cov{j} y_{uv} & \text{if \(u = \source{}\)} \\
      \cov{i} y_{uv} & \text{if \(v = \sink{}\)} \\
      \min\Set*{\cov{i}, \cov{j}} y_{uv} & \text{otherwise}
    \end{cases} &
    \begin{split}
      \forall (u, v) \in A \\
      i = vcontig(u) \\
      j = vcontig(v)
    \end{split} \label{pbf_iterbin:milp:cst:flow:arc_flow_at_most_coverage} % chktex 25
  \end{align}
\end{Constraint}

The cumulative flow through a contig \(i \in \Contigs{}\) cannot exceed its coverage:
\begin{Constraint}
  \begin{equation}
    inflow(i) \leq \cov{i} \quad \forall i \in \Contigs{} \label{pbf_iterbin:milp:cst:flow:inflow_at_most_coverage} % chktex 25
  \end{equation}
\end{Constraint}
Where \[
  inflow(i) = \sum_{(u, v) \in A} f_{uv} + \sum_{(u, \rev{v}) \in A} f_{u\rev{v}} \quad (v, \rev{v}) = contigv(i)
\]

The cumulative flow into a contig \(i\) should be equal to the cumulative flow out of it. Same for the reverse \(\rev{i}\):
\begin{Constraint}
  \begin{equation}
    \sum_{(u, v) \in A} f_{u v} = \sum_{(v, w) \in A} f_{v w} \quad \forall v \in V \setminus \Set*{\source{}, \sink{}} \label{pbf_iterbin:milp:cst:flow:flow_conservation} % chktex 25
  \end{equation}
\end{Constraint}

The total flow value \(F\) equals to the flow out of \(\source{}\) and into \(\sink{}\):
\begin{Constraint}
  \begin{align}
    F & = \sum_{(\source{}, v) \in A} f_{\source{} v} \label{pbf_iterbin:milp:cst:flow:total_flow_source} \\ % chktex 25
    F & = \sum_{(v, \sink{}) \in A} f_{v \sink{}} \label{pbf_iterbin:milp:cst:flow:total_flow_sink} % chktex 25
  \end{align}
\end{Constraint}

The total flow is strictly positive:
\begin{Constraint}
  \begin{equation}
    F \geq \epsilon_F \quad \epsilon_F > 0 \label{pbf_iterbin:milp:cst:flow:total_flow_strictly_positive} % chktex 25
  \end{equation}
\end{Constraint}

An active arc has a strictly positive flow:
\begin{Constraint}
  \begin{equation}
    y_a \leq \frac{1}{\epsilon_F} f_a \quad \forall a \in A \label{pbf_iterbin:milp:cst:flow:arc_flow_strictly_positive} % chktex 25
  \end{equation}
\end{Constraint}

An active arc has a flow at least \(F\).
\begin{Constraint}
  \begin{align}
    F_a & \geq F - (1 - y_a) \max_{i \in \ContigSeeds{}}\Set{\cov{i}} & \forall a \in A \label{pbf_iterbin:milp:cst:flow:arc_flow_at_least_total_flow_1} \\ % chktex 25
    F_a & \leq F & \forall a \in A \label{pbf_iterbin:milp:cst:flow:arc_flow_at_least_total_flow_2} \\ % chktex 25
    F_a & \leq f_a & \forall a \in A \label{pbf_iterbin:milp:cst:flow:arc_flow_at_least_total_flow_3} % chktex 25
  \end{align}
\end{Constraint}
\begin{infobox}
  These constraints do not force the arc flows to be a multiple of \(F\), see~\zcref[S]{proposition:inflow_is_not_multiple_of_total_flow}.
\end{infobox}
\begin{missingproofbox}
  The above constraints minimize the number of out link-arcs for each fragment.
\end{missingproofbox}
\begin{questionbox}
  What is the meaning for a fragment to have a cumulative flow that is not a multiple of \(F\)?
  By keeping the flow real, can we smartly force the cumulative flow to be a multiple of \(F\)?
\end{questionbox}

\paragraph{Connected component constraints}

Exactly one link-arc outs of \(\source{}\) is part of the solution (necessary to ensure the solution induced subgraph has only one connected component):
\begin{Constraint}
  \begin{equation}
    \sum_{(\source{}, v) \in A} y_{\source{} v} = 1 \label{pbf_iterbin:milp:cst:ccomp:one_outgoing_arc_from_source} % chktex 25
  \end{equation}
\end{Constraint}

A positive flow implies the two incident vertices are in the component:
\begin{Constraint}
  \begin{equation}
    1 - y_{uv} \geq
    \begin{cases}
      1 - x_v & \text{if \(u = \source{}\)} \\
      x_u - 1 & \text{if \(v = \sink{}\)} \\
      x_u - x_v & \text{otherwise}
    \end{cases}
    \quad \forall (u, v) \in A
    \label{pbf_iterbin:milp:cst:ccomp:positive_flow_incident_vertices_in_component} % chktex 25
  \end{equation}
\end{Constraint}
%
\begin{questionbox}
  potentially redundant with \zcref[S]{pbf_iterbin:milp:cst:dvar:active_vertex_active_incoming_arcs}
\end{questionbox}

The cumulative depth of the subtree in the exploration tree from the source equals at least the number of active vertices minus 1.
\begin{Constraint}
  \begin{align}
    \alpha + \sum_{(\source{}, w) \in A} \beta_{sw} \leq 1 \label{pbf_iterbin:milp:cst:ccomp:depth_of_tree_source} % chktex 25
  \end{align}
\end{Constraint}

Two incident arcs in the exploration tree are distanced by one.
\begin{Constraint}
  \begin{align}
    \sum_{(v, w) \in A} \beta_{vw} - \sum_{(u, v) \in A} \beta_{uv} \leq 1 \quad \forall v \in V \setminus \Set*{\source{}}
    \label{pbf_iterbin:milp:cst:ccomp:depth_of_tree_incident_arcs} % chktex 25
  \end{align}
\end{Constraint}

An arc participates in the exploration tree implies the arc is active.
\begin{Constraint}
  \begin{align}
    \beta_a \geq - y_a |V| \quad \forall a \in A
    \label{pbf_iterbin:milp:cst:ccomp:tree_arc_active} % chktex 25
  \end{align}
\end{Constraint}

The number of active vertices equals \(\alpha{}\) (SAT constraint):
\begin{Constraint}
  \begin{align}
    2 + \sum_{v \in V \setminus \Set*{\source{}, \sink{}}} x_v = \alpha
    \label{pbf_iterbin:milp:cst:ccomp:number_of_active_vertices} % chktex 25
  \end{align}
\end{Constraint}

\paragraph{GC constraints}
%
The following constraints define the GC content labelling of a bin.

\phantom{text}

Only one GC content interval \(b \in K\) labels the bin:
\begin{Constraint}
  \begin{equation}
    \sum_{b \in K} GC_b = 1
    \label{pbf_iterbin:milp:cst:gc:exactly_one_gc_content_interval} % chktex 25
  \end{equation}
\end{Constraint}

For each fragment \(i \in \Fragments{}\), for each GC content interval \(b \in K\), \(\contiggc{i}{b} = 1\) if and only if fragment \(i\) is active and the bin is labelled by the GC content interval \(b\):
\begin{Constraint}
  \begin{align}
    \contiggc{i}{b} & \leq x_{i_t} + x_{i_h} & \forall (i, b) \in \Fragments{} \times K \label{pbf_iterbin:milp:cst:gc:active_fragment_gc_1} \\ % chktex 25
    \contiggc{i}{b} & \leq GC_b & \forall (i, b) \in \Fragments{} \times K \label{pbf_iterbin:milp:cst:gc:active_fragment_gc_2} \\ % chktex 25
    \contiggc{i}{b} & \geq x_{i_t} + GC_b - 1 & \forall (i, b) \in \Fragments{} \times K \label{pbf_iterbin:milp:cst:gc:active_fragment_gc_3} \\ % chktex 25
    \contiggc{i}{b} & \geq x_{i_h} + GC_b - 1 & \forall (i, b) \in \Fragments{} \times K \label{pbf_iterbin:milp:cst:gc:active_fragment_gc_4} % chktex 25
  \end{align}
\end{Constraint}

For each fragment \(i \in \Fragments{}\), for each GC content interval \(b \in K\), \(\inflowgc{i}{b}\) is the flow into fragment \(i\) when \(b\) is the chosen GC content interval:
\begin{Constraint}
  \begin{align}
    \inflowgc{i}{b} & \leq \cov{i} \contig{i} & \forall (i, b) \in \Fragments{} \times K \label{pbf_iterbin:milp:cst:gc:inflowgc_1}  \\ % chktex 25
    \inflowgc{i}{b} & \leq \cov{i} GC_b & \forall (i, b) \in \Fragments{} \times K \label{pbf_iterbin:milp:cst:gc:inflowgc_2}  \\ % chktex 25
    \inflowgc{i}{b} & \geq inflow(i)  - (2 - \contig{i} - GC_b) \cov{i} & \forall (i, b) \in \Fragments{} \times K \label{pbf_iterbin:milp:cst:gc:inflowgc_3} % chktex 25
  \end{align}
\end{Constraint}

\paragraph{Plasmid constraints}

The cumulative fragment length is above a given threshold:
%
\begin{Constraint}
  \begin{equation}
    \sum_{i \in \ContigSeeds{}} |i| \geq L \quad L \geq 0 \label{pbf_iterbin:milp:cst:misc:min_cumulative_fragment_length} % chktex 25 chktex 35
  \end{equation}
\end{Constraint}
%
Typically, we choose \(L = 1000\).