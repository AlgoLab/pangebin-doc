\subsubsection{The Maximum GC Score problem \MGC{}}\label{sec:pbf_iterbin:decomp:mgc}

The second binning stage consists in finding a flow maximizing the GC scores, by fixing the coverage score flow.
We refer to this subproblem as the \enquote{Maximum GC Score} problem (\MGC{}).
In the following we describe the MILP model.

\begin{table}[h!]
  \centering
  \tablecaption{\MGC{} MILP constraints}{}\label{tab:decomp:mgc:cst}
  \begin{tabular}{@{}ll@{}}
    \toprule
    \tabhtxt{Category} & \tabhtxt{Constraints} \\
    \midrule
    \multicolumn{2}{@{}l@{}}{Constraints in \zcref[S]{tab:decomp:mcf:cst}} \\
    \addlinespace
    GC & \zcref[S]{%
      pbf_iterbin:milp:cst:gc:exactly_one_gc_content_interval,%
      pbf_iterbin:milp:cst:gc:active_contig_gc_1,%
      pbf_iterbin:milp:cst:gc:active_contig_gc_2,%
      pbf_iterbin:milp:cst:gc:active_contig_gc_3,%
      pbf_iterbin:milp:cst:gc:active_contig_gc_4%
    } \\
    \MGC{} dedicated & \zcref[S]{pbf_iterbin:decomp:mgc:cst:fix_mcf_obj} \\
    \bottomrule
  \end{tabular}
\end{table}

In addition to the constraints in \zcref[S]{tab:decomp:mgc:cst}, we constraint the coverage score to be near to the optimal coverage score \(\Phi{}\) previously found in \MCF{}:
\begin{Constraint}
  \begin{equation}
    \Phi - (1 - \gamma_1) | \Phi | \leq CoverageScore \quad \gamma_1 \in (0.5, 1] % chktex 9
    \label{pbf_iterbin:decomp:mgc:cst:fix_mcf_obj} % chktex 25
  \end{equation}
\end{Constraint}

\begin{definition}{\MGC{} MILP objective function}{pbf_iterbin:decomp:mgc:obj}
  The objective function aims to maximize the GC score:
  \begin{Objective}
    \begin{equation}
      \max ~ GCScore
      \label{pbf_iterbin:decomp:mgc:obj:max_gc_probability_score} % chktex 25 % chktex 35
    \end{equation}
  \end{Objective}
  where
  \begin{equation*}
    GCScore = \sum_{\substack{
        i \in \Contigs{} \\
        k \in K
    }} \gcscore{i}{k} \contiggc{i}{k}
  \end{equation*}

  \begin{notebox}
    In the opposite of the other \texttt{decomp} models, in \MGC{} the GC score does not depend on the contig coefficient \fzeta{i}.
  \end{notebox}
\end{definition}