\subsubsection{The Maximum GC Score problem \MGC{}}\label{sec:pbf_iterbin:decomp:mgc}

The second binning stage consists in finding a flow maximizing the GC scores, by fixing the coverage score flow.
We refer to this subproblem as the \enquote{Maximum GC Score} problem (\MGC{}).
In the following we describe the MILP model.

\begin{table}[h!]
  \centering
  \tablecaption{\MGC{} MILP constraints}{}\label{tab:decomp:mgc:cst}
  \begin{tabular}{@{}ll@{}}
    \toprule
    \tabhtxt{Category} & \tabhtxt{Constraints} \\
    \midrule
    \multicolumn{2}{@{}l@{}}{Constraints in \Cref{tab:decomp:mcf:cst}} \\
    \addlinespace
    GC & \Cref{%
      pbf_iterbin:milp:cst:gc:exactly_one_gc_content_interval,%
      pbf_iterbin:milp:cst:gc:active_fragment_gc_1,%
      pbf_iterbin:milp:cst:gc:active_fragment_gc_2,%
      pbf_iterbin:milp:cst:gc:active_fragment_gc_3,%
      pbf_iterbin:milp:cst:gc:active_fragment_gc_4%
    } \\
    \MGC{} dedicated & \Cref{pbf_iterbin:decomp:mgc:cst:fix_mcf_obj} \\
    \bottomrule
  \end{tabular}
\end{table}

In addition to the constraints in \Cref{tab:decomp:mgc:cst}, we constraint the coverage score to be near to the optimal coverage score \(\Phi{}\) previously found in \MCF{}:
\begin{equation}
  \Phi - (1 - \gamma_1) | \Phi | \leq CoverageScore \quad \gamma_1 \in (0.5, 1] % chktex 9
  \cstlabel{pbf_iterbin:decomp:mgc:cst:fix_mcf_obj} % chktex 25
\end{equation}

\begin{definition}{\MGC{} MILP objective function}{pbf_iterbin:decomp:mgc:obj}
  The objective function aims to maximize the GC score:
  \begin{equation}
    \max ~ GCScore
    \objlabel{pbf_iterbin:decomp:mgc:obj:max_gc_probability_score} % chktex 25 % chktex 35
  \end{equation}
  where \(
    \displaystyle GCScore = \sum_{\substack{
        i \in \Fragments{} \\
        b \in K
    }} \gcscore{i}{b} \fraggc{i}{b}%
  \).

  \begin{notebox}
    In the opposite of the other \texttt{decomp} models, in \MGC{} the GC score does not depend on the fragment coefficient \fzeta{i}.
  \end{notebox}
\end{definition}