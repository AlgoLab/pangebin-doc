\subsubsection{The Maximum Plasmidness Score problem \MPS{}}\label{sec:pbf_iterbin:decomp:mps}

The third binning stage consists in finding a flow maximizing the plasmidness scores, by fixing both the coverage flow and the GC score.
We refer to this subproblem as the \enquote{Maximum Plasmidness Score} problem (\MPS{}).
In the following we describe the MILP model.

\begin{table}[h!]
  \centering
  \tablecaption{\MPS{} MILP constraints}{}\label{tab:decomp:mps:cst}
  \begin{tabular}{@{}ll@{}}
    \toprule
    \tabhtxt{Category} & \tabhtxt{Constraints} \\
    \midrule
    \multicolumn{2}{@{}l@{}}{Constraints in \zcref[S]{tab:decomp:mgc:cst}} \\
    \addlinespace
    \MPS{} dedicated & \zcref[S]{pbf_iterbin:decomp:mps:cst:fix_mgc_obj} \\
    \bottomrule
  \end{tabular}
\end{table}

The total GC score value must be near to the total GC score best value \(\Psi{}\) previously found in \MGC{}:
\begin{Constraint}
  \begin{equation}
    \gamma_2 \Psi \leq GCScore \quad \gamma_2 \in (0.5, 1] % chktex 9
    \label{pbf_iterbin:decomp:mps:cst:fix_mgc_obj} % chktex 25
  \end{equation}
\end{Constraint}

\begin{definition}{\MPS{} MILP objective function}{pbf_iterbin:decomp:mps:obj}
  \begin{newfeatbox}
    While in the original PlasBin-flow method the plasmidness is positive, here the it either acts as a bonus or a penalty (\( \plm{i} \in [-1, 1] \)).
  \end{newfeatbox}
  %
  The objective function aims to maximize the plasmidness scores:
  \begin{Objective}
    \begin{equation}
      \max ~ PlasmidnessScore
      \label{pbf_iterbin:decomp:mps:obj:max_plasmidness_score} % chktex 25 % chktex 35
    \end{equation}
  \end{Objective}
  where
  \begin{equation*}
    PlasmidnessScore = \sum_{i \in \Contigs{}} \fzeta{i} \plm{i} \contig{i}
  \end{equation*}
\end{definition}
