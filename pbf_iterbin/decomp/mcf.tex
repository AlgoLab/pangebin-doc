\subsubsection{The Maximum Coverage-likelihood Flow problem \MCF{}}\label{sec:pbf_iterbin:decomp:mcf}

The first binning stage consists in finding a flow maximizing the use of fragment coverage.
We refer to this subproblem as the \enquote{Maximum Coverage-likelihood Flow} problem (\MCF{}).
As a coarse-grain strategy, it consists in finding a connected component explaining the coverages, large enough to contain all the fragments of the in-building solution bin.
Indeed, passing through a loop or a cycle will not change the flow value. Only maximizing the flow can lead to use a minimum of fragments.
To overcome this bias, we score the use of the fragment coverages and maximize their total.
Below we describe the MILP model.

\begin{table}[h!]
  \centering
  \tablecaption{\MCF{} MILP constraints}{}\label{tab:decomp:mcf:cst}
  \begin{tabular}{@{}ll@{}}
    \toprule
    \tabhtxt{Category} & \tabhtxt{Constraints} \\
    \midrule
    Decision & \Cref{%
      pbf_iterbin:milp:cst:dvar:active_extremity_implies_active_fragment,%
      pbf_iterbin:milp:cst:dvar:active_fragment_implies_active_extremity,%
      pbf_iterbin:milp:cst:dvar:active_link_arc_active_fragments,%
      pbf_iterbin:milp:cst:dvar:active_vertex_active_incoming_arcs%
    } \\
    Flow & \Cref{%
      pbf_iterbin:milp:cst:flow:arc_flow_at_most_coverage,%
      pbf_iterbin:milp:cst:flow:inflow_at_most_coverage,%
      pbf_iterbin:milp:cst:flow:inflow_equals_outflow_forward,%
      pbf_iterbin:milp:cst:flow:inflow_equals_outflow_reverse,%
      pbf_iterbin:milp:cst:flow:total_flow_source,%
      pbf_iterbin:milp:cst:flow:total_flow_sink,%
      pbf_iterbin:milp:cst:flow:arc_flow_at_least_total_flow_1,%
      pbf_iterbin:milp:cst:flow:arc_flow_at_least_total_flow_2,%
      pbf_iterbin:milp:cst:flow:arc_flow_at_least_total_flow_3%
    } \\
    Connected component & \Cref{%
      pbf_iterbin:milp:cst:ccomp:one_outgoing_arc_from_source,%
      pbf_iterbin:milp:cst:ccomp:positive_flow_incident_vertices_in_component,%
      pbf_iterbin:milp:cst:ccomp:depth_of_tree_source,%
      pbf_iterbin:milp:cst:ccomp:depth_of_tree_incident_arcs,%
      pbf_iterbin:milp:cst:ccomp:tree_arc_active,%
      pbf_iterbin:milp:cst:ccomp:number_of_active_vertices%
    } \\
    GC & --- \\
    Miscellaneous & \Cref{%
      pbf_iterbin:milp:cst:misc:min_cumulative_fragment_length%
    } \\
    \bottomrule
  \end{tabular}
\end{table}

\begin{definition}{\MCF{} MILP objective function}{pbf_iterbin:decomp:mcf:obj}
  \begin{newfeatbox}
    Here we maximize the coverage scores, not only the coverage flow.
  \end{newfeatbox}
  The objective function aims to maximize the total coverage scores:
  \begin{equation}
    \max ~ CoverageScore
    \objlabel{pbf_iterbin:decomp:mcf:obj:max_coverage_score} % chktex 25 % chktex 35
  \end{equation}
  where \(
    \displaystyle CoverageScore = %
    \sum_{i \in \Fragments{}} \fzeta{i} \parenth*{ inflow(i) - (\cov{i}\frag{i} - inflow(i)) }
  \) and \( \displaystyle \fzeta{i} = \frac{ |i| }{ \max_{j \in \Fragments{}} |j| } \).

  \begin{notebox}
    Before we had \(
      \displaystyle CoverageScore = %
      \sum_{i \in \Fragments{}} \parenth*{2 \frac{inflow(i)}{\cov{i}} - 1}
    \)
  \end{notebox}

  \begin{ideabox}
    You can get inspired from the future multi-flow ideas and define the coverage score to be:
    \[
      CoverageScore = \sum_{
        \substack{
          i \in \Fragments{} \\
          \plm{i} > 0
        }
      } |i| \parenth*{ inflow(i) - \cov{i} }
    \]
    where we focus on the use of positive plasmidness fragments, and try to minimize the penalty of not using their coverage.
    Multipliying by the fragment length keep the things robust against the fragmentation of the sequences (good property for the pan-assembly fragments).
  \end{ideabox}
\end{definition}