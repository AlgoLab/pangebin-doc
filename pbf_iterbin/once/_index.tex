\subsection{Approach \texttt{once}}\label{sec:pbf_iterbin:once}

The \texttt{once} approach tackles the binning of the contigs with only one optimization problem (\MGCLB{} problem, \zcref[S]{sec:pbf_iterbin:once:mgclb}) where the different bin properties are involved into only one objective function.
This approach is the nearer to the PlasBin-flow approach comparing to \texttt{decomp} and \texttt{binlab}.

\zcref[S]{algo:pbf_iterbin:once} summarizes the \texttt{once} approach.

\begin{tcbalgo}{\texttt{once} approach}{pbf_iterbin:once}
  \begin{algorithmic}[1]
    \Require{%
      A network graph \(N\) as in \zcref[S]{definition:pbf_iterbin:network_graph}.
    }
    \Ensure{%
      Extract the best bin from the network graph.
    }
    \Function{find\_best\_bin}{\( N \)}
    \State{} Init \MGCLB{} and solve it
    \If{\MGCLB{} is unfeasible}
    \State{} \Return{}unfeasible message
    \EndIf{}
    \State{} \Return{}\MGCLB{} bin result (defined by the MILP solution variables induced subgraph)
    \EndFunction{}
  \end{algorithmic}
\end{tcbalgo}

\subsubsection{The Maximum GC-Labelled Bin problem \MGCLB{}}\label{sec:pbf_iterbin:once:mgclb}

The first binning stage consists in finding a flow maximizing the use of fragment coverage and the plasmidness.
We refer to this subproblem as the \enquote{Maximum GC-Labelled Bin} problem (\MGCLB{}).
In the opposite of \texttt{decomp} and \texttt{binlab}, here we mix the GC scores with the other bin properties.
In order  balance the weight of the different semantic terms, we multiply by the incoming flow.
The more a fragment participates in terms of its coverage in the bin, the more its properties are amplified.
The length of the fragments also enable to balance the usage of the fragments in the bin.
It is also robust to the fragmentation (linear decomposition).
Below we describe the MILP model.

\begin{table}[h!]
  \centering
  \tablecaption{\MGCLB{} MILP constraints}{%
    What is new here comparing to the two previous approaches is we need to define variables \(\inflowgc{i}{b}\) for all fragments \(i \in \Fragments{}\) and GC content interval \(b \in K\).
    Also, we do not need to define the \(\fraggc{\Fragments}{K}\) binary variables.
  }\label{tab:once:mgclb:cst}
  \begin{tabular}{@{}ll@{}}
    \toprule
    \tabhtxt{Category} & \tabhtxt{Constraints} \\
    \midrule
    Decision & \zcref[S]{%
      pbf_iterbin:milp:cst:dvar:active_vertex_implies_active_contig,%
      pbf_iterbin:milp:cst:dvar:active_contig_implies_active_vertex,%
      pbf_iterbin:milp:cst:dvar:active_arc_active_vertices,%
      pbf_iterbin:milp:cst:dvar:active_vertex_active_incoming_arcs%
    } \\
    Flow & \zcref[S]{%
      pbf_iterbin:milp:cst:flow:arc_flow_at_most_coverage,%
      pbf_iterbin:milp:cst:flow:inflow_at_most_coverage,%
      pbf_iterbin:milp:cst:flow:flow_conservation,%
      pbf_iterbin:milp:cst:flow:total_flow_source,%
      pbf_iterbin:milp:cst:flow:total_flow_sink,%
      pbf_iterbin:milp:cst:flow:total_flow_strictly_positive,%
      pbf_iterbin:milp:cst:flow:arc_flow_at_least_total_flow_1,%
      pbf_iterbin:milp:cst:flow:arc_flow_at_least_total_flow_2,%
      pbf_iterbin:milp:cst:flow:arc_flow_at_least_total_flow_3%
    } \\
    Connected component & \zcref[S]{%
      pbf_iterbin:milp:cst:ccomp:one_outgoing_arc_from_source,%
      pbf_iterbin:milp:cst:ccomp:positive_flow_incident_vertices_in_component,%
      pbf_iterbin:milp:cst:ccomp:depth_of_tree_source,%
      pbf_iterbin:milp:cst:ccomp:depth_of_tree_incident_arcs,%
      pbf_iterbin:milp:cst:ccomp:tree_arc_active,%
      pbf_iterbin:milp:cst:ccomp:number_of_active_vertices%
    } \\
    GC & \zcref[S]{%
      pbf_iterbin:milp:cst:gc:exactly_one_gc_content_interval,%
      pbf_iterbin:milp:cst:gc:inflowgc_1,%
      pbf_iterbin:milp:cst:gc:inflowgc_2,%
      pbf_iterbin:milp:cst:gc:inflowgc_3%
    } \\
    Miscellaneous & \zcref[S]{%
      pbf_iterbin:milp:cst:misc:min_cumulative_contig_length%
    } \\
    \bottomrule
  \end{tabular}
\end{table}

\begin{definition}{\MGCLB{} MILP objective function}{pbf_iterbin:once:mgclb:obj}
  The objective function aims to maximize the total coverage scores:
  \begin{Objective}
    \begin{equation}
      \max ~ CoveragePenaly + PlasmidnessScore + GCScore
      \label{pbf_iterbin:once:mgclb:obj:max_coverage_score} % chktex 25 % chktex 35
    \end{equation}
  \end{Objective}
  where:
  \begin{align*}
    CoveragePenaly &= %
    \sum_{i \in \Fragments{}} |i| \parenth*{ inflow(i) - \cov{i}\frag{i} } \\
    PlasmidnessScore &= %
    \sum_{i \in \Fragments{}} |i| \plm{i} inflow(i) \\
    GCScore &= %
    \sum_{\substack{
        i \in \Fragments{} \\
        b \in K
    }} |i| \gcscore{i}{b} \inflowgc{i}{b}
  \end{align*}

  \begin{ideabox}
    Perhaps we should use the penalty coverage on only fragments with positive plasmidness.
  \end{ideabox}
\end{definition}