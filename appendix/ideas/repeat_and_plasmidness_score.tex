\subsection{Considering repeats for the plasmidness score}

The plasmidness score do not consider the use of a fragment several time.
From~\Cref{proposition:repeat_has_several_incoming_arcs}, and the constraints forcing the active arc flow to be at least equals to \(F\), we would like to model the number of time a fragment is used according to the sum of incoming flows.
However, the active arc flows are not necessary multiple of \(F\) (\Cref{proposition:inflow_is_not_multiple_of_total_flow}).

One would want to propose to weigth the \(\plm{i}\) with the incoing flow normalized by \(F\):
\[
  WeigthedPlasmidnessScore = \sum_{i \in \Fragments{}} \plm{i} \frac{inflow(i)}{F}
\]
But this expression is non-linear because \(F\) is a variable.

One idea is to aproximate \(F\).
For this purpose, we can use constraint~\Cref{pbf_iterbin:decomp:mgc:cst:fix_mcf_obj} forcing \(F\) to be in an interval given by the objective value of \MCF{}, and propose a constant \(F'\):
\[
  F' = \frac{ (1 + \gamma_1) \Phi }{2}
\]
We can now propose a new objective function:

\begin{definition}{\MPS{} MILP alternative objective function}{pbf_iterbin:decomp:mps:obj_weigthed}
  The objective function aims to maximize the weighted plasmidness scores:
  \begin{equation}
    \max ~ WeightedPlasmidnessScore
    \objlabel{mps:obj:max_weighted_plasmidness_score} % chktex 25 % chktex 35
  \end{equation}
  where \(\displaystyle WeigthedPlasmidnessScore = \sum_{i \in \Fragments{}} 2 (\plm{i} - 0.5) \frac{inflow(i)}{F'} \).
\end{definition}
