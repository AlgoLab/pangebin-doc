\subsection{Fine-tuning the MILP formulations}

\begin{warningbox}
  This part is in development
\end{warningbox}

\begin{todobox}
  Describe fine-tuning for:
  \begin{CheckList}{Task}
    \Task{open}{The GC content penalty}
    \Task{open}{The plasmidness}
  \end{CheckList}
\end{todobox}

Because of the fragmentation, for a given attribute, we do not have a strict attribute-equivalence between each contig and its corresponding fragment path in the pan-assembly graph.
Formally, if \(\omega{}\) is an attribute defined either for the contigs and the fragments, \(\forall c \in \Contigs{}\) the statement \(\omega_c = \sum_{i \in \Fragments{}(c)} \omega_i\) does not necessarily hold.
We interpret this issue differentially in the GC content case (\MGC{} problem) and the plasmidness case (\MPS{} problem).
\zcref[S]{sec:pbf_iterbin:decomp:mgc:fine_tuned,sec:pbf_iterbin:decomp:mps:fine_tuned} respectively detail the two issues and propose corrective approaches.

Given a contig \(c \in \Contigs{}\), we say it is active if and only if either for its forward or its reverse the link-arcs defining them in the network are active.

\subsubsection{Correct the GC content score}\label{sec:method:mgc:fine_tuned}

The calculation of the GC score for a fragment \(i \in \Fragments{}\) does not rely to the belonging of \(i\) in its contigs \(c \in \Contigs{}(i)\).

\begin{questionbox}
  The plasmidness correction is simpler (see~\Cref{sec:method:mps:fine_tuned}) because in we just want to add a bonus.
  Here the question is: if the GC scores of the contigs have to be considered, can we choose one or a subset of contigs from those active to correct the GC content score?
  \begin{itemize}
    \item The contig with the highest absolute GC score difference?
    \item The contig with the highest GC Penalties probability score (bonus)?
    \item The contig with the lowest GC Penalties probability score (penalty)?
  \end{itemize}
  On related issue here is: should we choose only one difference, or a subset?
  The sum of relative GC score differences?

  \begin{todobox}
    Choose one and fix what follows
  \end{todobox}
\end{questionbox}

\begin{fixmebox}
  Because of the new attributes definitions, the corrections introductions and definitions must be adapted.
\end{fixmebox}

\begin{definition}{GC score bonus}{frag_gc_bonus}
  \begin{fixmebox}
    The GC penalty has became a GC prob.\ score: adapt the following definition.
  \end{fixmebox}
  Given a contig \(c \in \Contigs{}\) and a GC content interval \(b \in K\), we define \(\dcgcscore{c}{b}\) as the as a positive GC score bonus for the subcontigs of \(c\):
  \[
    \dcgcscore{c}{b} = \max \Set*{
      0,
      \sum_{\substack{
          i \in \Fragments{}(c) \\ i \text{ a subcontig}
      }}
      \parenth*{
        \frac{|i|}{|c|}\gcscore{c}{b} - \gcscore{i}{b}
      }
    }
  \]
  The bonus for the share depends on which contigs they belong.
  Given a share \(j \in \Fragments{}\) and a GC content interval \(b \in K\), we define \(\dsgcscore{j}{c}{b}\) as the positive difference between the GC penalty from the normalization of the fragment penalty according to the contig and the fragment penalty computed independently of the contigs it belongs:
  \[
    \dsgcscore{j}{c}{b} = \max \Set*{
      0,
      \frac{|j|}{|c|} \gcscore{c}{b} - \gcscore{j}{b}
    }
  \]
  \begin{questionbox}
    A real equivalence between a contig and the sequence of its fragments should imply the \enquote{bonus} to be possibly negative.
    How to argue in favour of a positive bonus rather than a real correction?

    Be aware that if the deltas can be negatives, then we must change the associated constraints
  \end{questionbox}
\end{definition}

\begin{ideabox}
  As for a contig \(c\),
  \[
    \gcscore{c}{b} \leq \dcgcscore{c}{b} + \sum_{\substack{
        j \in \Fragments{}(c) \\
        j \text{ is a share}
    }}\dsgcscore{j}{c}{b}
  \]
  If we don't want to count twice the share participations, we should define:
  \[
    \dcgcscore{c}{b} = \max \Set*{
      0,
      \gcscore{c}{b} - \sum_{i \in \Fragments{}(c)} \gcscore{i}{b}
    }
  \]
  and if a better contig is active, remove the share participation of the worst.

  \begin{questionbox}
    What should we not count twice a share participation, while if it is counted twice, it is because it is used several times (a repeat).

    Perhaps the best thing is just to add a bonus (or correct, negatively too)
  \end{questionbox}
\end{ideabox}

\begin{definition}{\MGC{} fine-tune variables}{mgc:milp:variables:fine_tune}
  \begin{itemize}
    \item \(x_c \in \Set{0, 1} \, \forall c \in \Contigs{}\) denoting whether the contig \(c\) is active or not.
    \item \(\ctggc{c}{b} \in \Set{0, 1} \, \forall (c, b) \in \Contigs{} \times K\) denoting whether the contig \(c\) participates in the solution and the plasmid GC content is in the interval \(b\) or not.
    \item \(\sgcbonus{j}{b} \in \Reals_{\geq 0} \, \forall \text{ share } j \in \Fragments{}, \forall b \in K\) corresponding to the best GC bonus over the GC content penalty for the share \(j\) when several of its contigs are active, given a GC content interval \(b\).
  \end{itemize}
\end{definition}

\begin{definition}{\MGC{} fine-tune constraint}{milp:mcf:constraints:fine_tune}
  A contig \(c \in \Contigs{}\), is active if and only if all the link-arcs defining \(c\) (or its reverse) are active:
  \begin{align}
    |A_\Links{}(c)|x_c & \leq \sum_{a \in A_\Links{}(c)} y_a \\
    |A_\Links{}(c^-)|x_c & \leq \sum_{a \in A_\Links{}(c^-)} y_a
  \end{align}

  \begin{questionbox}
    Depending on the question (bonus, penalty, relative correction), modelling the \enquote{if and only if} statement may be required.
  \end{questionbox}

  For each contig \(c \in \Contigs{}\), for each GC content interval \(b \in K\), \(\ctggc{c}{b} = 1\) if and only if contig \(c\) is active and the \(b\) is the solution GC content interval:
  \begin{align}
    \ctggc{c}{b} & \leq x_c & \forall (c, b) \in \Contigs{} \times K \\
    \ctggc{c}{b} & \leq GC_b & \forall (c, b) \in \Contigs{} \times K \\
    \ctggc{c}{b} & \geq x_c + GC_b - 1 & \forall (c, b) \in \Contigs{} \times K
  \end{align}

  For each share \(j \in \Fragments{}\), for each GC content interval \(b \in K\), \(\sgcbonus{j}{b}\) is the best correction \(\dsgcscore{j}{c}{b}\) among the active contigs in \(\Contigs(j)\):
  \begin{equation}
    \sgcbonus{j}{b} \leq \sum_{\substack{
        d \in \Contigs(j) \\ \dsgcscore{j}{d}{b} \\ \geq \dsgcscore{j}{c}{b}
    }} \dsgcscore{j}{d}{b} \ctggc{d}{b} \quad
    \begin{array}[t]{@{}l@{}}
      \forall (j, c, b) \in \Fragments{} \times \Contigs{}(j) \times K \\
      j \text{ is a share}
    \end{array}
  \end{equation}

\end{definition}

\begin{definition}{\MGC{} objective function with correction}{milp:mcf:objective:fine_tune}
  \begin{fixmebox}
    Depending of the question (bonus, penalty, relative correction), the objective function must be adapted.
  \end{fixmebox}
  The objective function aims to maximize the corrected GC score:
  \begin{equation}
    \max ~ GCCorrectedScore
  \end{equation}
  Where
  \begin{align}
    \begin{split}
      GCCorrectedScore = & \sum_{\substack{
          i \in \Fragments{} \\
          b \in K
      }} \gcprob{i}{b} \fraggc{i}{b} \\
      & - \sum_{\substack{
          c \in \Contigs{} \\
          b \in K
      }} \dcgcscore{c}{b}\ctggc{c}{b} \\
      & - \sum_{\substack{
          \text{share } j \in \Fragments{} \\
          b \in K
      }} \sgcbonus{j}{b}
    \end{split}
  \end{align}
\end{definition}

\subsection{The Maximum Plasmidness Score problem \MPS{}}\label{sec:method:mps}

The third binning stage consists in finding a flow maximizing the plasmidness scores, by fixing both the coverage flow and the GC probability score.
We refer to this subproblem as the \enquote{Maximum Plasmidness Score} problem (\MPS{}).
In the following we describe the MILP model.

\begin{definition}{\MPS{} MILP variables}{milp:mps_variables}
  We complete \Cref{definition:milp:mgc_variables} with the following variables:
  \(\frag{i} \in [0, 1] \, \forall i \in \Fragments{}\) denoting whether fragment \(i\) is active or not. With the constraint it acts as a binary.
\end{definition}

\begin{definition}{\MGC{} MILP constraints}{milp:mps_constraints}
  To the constraints in \Cref{definition:milp:mgc_constraints} we add the following:

  The total GC probability score value must be near to the total GC probability score value \(\Psi{}\) found in \MGC{}:
  \begin{equation}
    \gamma_2 \Psi \leq GCProbabilityScore \quad \gamma_2 \in (0.5, 1]
  \end{equation}

  The fragment is active if at least one of its extremities is active:
  \begin{align}
    \frag{i} \leq x_{i_t} + x_{i_h} \quad \forall i \in \Fragments{} \\
    x_{i_t} \leq \frag{i} \quad \forall i \in \Fragments{} \\
    x_{i_h} \leq \frag{i} \quad \forall i \in \Fragments{}
  \end{align}
\end{definition}

\begin{definition}{\MGC{} MILP objective function}{milp:mps_objective}
  \begin{newfeatbox}
    Here the plasmidness either act as a bonus if they are more than 0.5, or a penalty if they are less than 0.5.
  \end{newfeatbox}
  The objective function aims to maximize the plasmidness scores:
  \begin{equation}
    \max ~ PlasmidnessScore
  \end{equation}
  where \(\displaystyle PlasmidnessScore = \sum_{i \in \Fragments{}} 2 (\plm{i} - 0.5)\frag{i}\).
\end{definition}
