\section*{Prelude}\label{sec:prelude}
\addcontentsline{toc}{section}{\nameref{sec:prelude}}

\subsection*{Aims}\label{sec:prelude:aims}
\addcontentsline{toc}{subsection}{\nameref{sec:prelude:aims}}

\begin{itemize}
  \item Comparing PlasBin-flow and Pangebin:
    \begin{itemize}
      \item with only one assembler as input (contig binning)
      \item with two assemblers as inputs (pangenome fragment binning)
    \end{itemize}
  \item Continue the idea of correcting the plasmidness for the fragments if we pass through an entire contig.
\end{itemize}

\begin{newfeatbox}
  Done since last times:

  \begin{CheckList}{Task}
    \Task{done}{Model a connected-flow in MILP to avoid iteratively resolving the problem until the solution subgraph has only one connected component. This answers one point of the PlasBin-flow discussion (\Cref{sec:method:mcf})}
    \Task{done}{Formaly define the panassembly network (\Cref{definition:panassembly_network})}
    \Task{done}{Simplify notations for the MILP}
  \end{CheckList}
\end{newfeatbox}

\subsection*{To-dos}\label{sec:prelude:to-dos}
\addcontentsline{toc}{subsection}{\nameref{sec:prelude:to-dos}}

\begin{todobox}
  Adapt PlasBin-flow to have two main indoors: one for SKESA/Unicycler asm graphs and one for both
\end{todobox}

\begin{todobox}
  Generalize gene density to plasmidness score (the plasmidness):

  \begin{CheckList}{Task}
    \Task{open}{Change attribute definition}
    \Task{open}{Give an example of how a plasidness score can be obtained}
    \Task{open}{Justify why optmizing the plasmidness is the last problem to solve in the hiearchy}
  \end{CheckList}
\end{todobox}