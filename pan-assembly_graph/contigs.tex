\subsection{Contigs in the pan-assembly graph}

Given a contig \(c\) in the set of contigs \(\Contigs{} = \Contigs{}_1 \cup \Contigs{}_2\) from the two assemblers (by \(c^-\) we denote the reverse).
In the pan-assembly graph \(PG\), there exists a unique sequence \(p_c\) of vertices in \(V\) corresponding to the contig \(c\).

\begin{notebox}
  Each sequence \(p_c\) can be decomposed as a fragment-edge --- link-edge alternate, beginning from and ending to a fragment-edge.
\end{notebox}

We denote by \(A(p_c)\) or \(A(c)\) the set of directed link-edges (link-arcs) composing contig \(c\).

\begin{notebox}
  By \(A(c^-)\) we denote the set of link-arcs composing the reverse of \(c\).
\end{notebox}

For a contig \(c \in \Contigs{}\), \(\Fragments{}(c) \subset \Fragments{}\) is the set of all fragments of \(c\).

\subsubsection{Contig properties}

Each contig is associated with three properties: its coverage, its plasmidness and a set of GC penalties depending on GC content intervals.

In the following, \(c \in \Contigs{}\) is a contig.

\begin{definition}{Contig coverage}{contig_coverage}
    By \(\cov{c} \in \Reals_{>0}\) we denote the coverage of \(c\).
    It corresponds to the average base coverage of the contig normalized by the median coverage of the contigs in the whole assembly.
\end{definition}

\begin{definition}{Contig GC probability score}{contig_GC_score}
    Let \(c \in \Contigs{}\) be a contig and \(b\) be a GC content interval.
    By \(\gcscore{c}{b} \in [-1, 1]\) we denote the GC probability score of the contig \(c\) according to the GC content interval \(b\):
    \[
    \gcscore{c}{b} = 2 \frac{Pr(n|b,l)Pr(b)}{\max\limits_{b' \in K}\Set*{Pr(n|b',l)Pr(b')}} - 1
    \]
    Where \(K\) is the set of GC content intervals, \(n\) is the number of GC in the contig of length \(l\), and \(Pr(n|b,l)\) is calculated as described in \cite{manePlasBinflowFlowbasedMILP2023}, Section 2.5.1.

      \begin{fixmebox}
        Adapt the correction constraints according to the new score definition.
    \end{fixmebox}
\end{definition}

\begin{newfeatbox}
    We do not normalize the probabilities by the sum of the \(P(b|n,l)\) over all \(b \in K\) as in \cite{manePlasBinflowFlowbasedMILP2023}. We normalize them according to the maximum among them.
\end{newfeatbox}

\begin{definition}{Contig gene density}{contig_gene_density}
    Let \(c \in \Contigs{}\) be a contig.
    By \(\gd{c} \in [0, 1]\) we denote its gene density:
    \[
    \gd{c} = \frac{ \ncovnuc{c} }{ |c| }
    \]
    Where \(\ncovnuc{c}\) is the number of nucleotides of contig \(c\) covered by gene mapping matches.
\end{definition}